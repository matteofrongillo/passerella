\documentclass{article}

\usepackage[fleqn]{amsmath}
\usepackage{amssymb}
\usepackage{hyperref}
\usepackage{url}
\usepackage{graphicx}
\usepackage{geometry}
\usepackage[italian]{babel}
\usepackage{enumitem}
\usepackage{parskip}
\usepackage{chemfig}
\usepackage{pdfpages}
\usepackage{xcolor}
\usepackage{tikz}
\usepackage{fancybox}
\usepackage{makecell}
\usepackage{pgfplots}
\usepackage{soul}
\usepackage{ulem}
\usepackage{wrapfig}
\usepackage{subcaption}
\usetikzlibrary{decorations.pathreplacing}
\pgfplotsset{compat=1.17}

\usepackage{frongillo}
\usepackage{stellar}

\geometry{
    a4paper,
    total={170mm, 257mm},
    left=20mm,
    top=20mm
}

\hypersetup{
    colorlinks=true,
    linkcolor=black,
    urlcolor=blue,
    pdftitle={Chimica}
}

% === TEXT ===
\title{\textbf{Matematica - Esame orale \\ Passerella 2023-24}}
\author{Matteo Frongillo, Paolo Bettelini}

\begin{document}

\maketitle
\tableofcontents
\pagebreak

\part{Analisi}

\section{Limiti}

\subsection{Concetto di limite di una funzione reale (interpretazione geometrica)}

\subsubsection{Determinare la definizione formale di limite}

\sdefinition{Limite di una funzione reale}{
    Data una funzione \(f:D\to \mathbb{R}\),
    il limte \[L=\lim_{x\to c}f(x)\] esiste se dato un
    \(\epsilon >0\) arbitrariamente piccolo, esiste un altro valore \(\delta >0\) tale che
    \[
        0<|x-c|<\delta
        \implies
        |f(x)-L|<\epsilon
    \]
}

\subsubsection{Disegnare geometricamente il limite:}\phantom{}

\begin{tikzpicture}[
    declare function={
        func(\x) = \x*\x / 10 + 1;
        LimX=4.25;
        Epsilon=0.7;
        Delta=0.5;
        LimY={func(LimX)};
        Width=7.5;
        Height=5;
    }
]
    \draw[->] (0, -0.25) -- (0, Height) node[right] {\(y\)};
    \draw[->] (-0.25, 0) -- (Width, 0) node[above] {\(x\)};

    \draw[-, dashed] (LimX, 0) -- (LimX, LimY);
    \draw[-, dashed] (0, LimY) node[left] {\(L\)} -- (LimX, LimY);

    \draw[-] (0, {LimY + Epsilon}) node[left] {\(L + \epsilon\)} -- (Width, {LimY + Epsilon});
    \draw[-] (0, {LimY - Epsilon}) node[left] {\(L - \epsilon\)} -- (Width, {LimY - Epsilon});

    \draw[-] ({LimX - Delta}, 0) node[below] {\(\scriptstyle c - \delta\)} -- ({LimX - Delta}, Height);
    \draw[-] ({LimX + Delta}, 0) node[below] {\(\scriptstyle c + \delta\)} -- ({LimX + Delta}, Height);

    \fill [green, opacity=0.2] (0,{LimY - Epsilon}) rectangle (Width, {LimY + Epsilon});
    \fill [red, opacity=0.2] ({LimX - Delta}, 0) rectangle ({LimX + Delta}, Height);

    \draw[domain=-0.25:6, smooth, variable=\x, blue, thick] plot ({\x}, {func(\x)});
\end{tikzpicture}

A parole possiamo dire che il limite esiste (ed è pari ad un valore che chiamiamo \(L\))
se la seguente condizione è soddisfatta:
\begin{enumerate}
    \item Scegliamo un \(\epsilon\) arbitrariamente piccolo e costruiamo l'intervallo \([L-\epsilon;L+\epsilon]\).
        Questo intervallo è un range attorno al valore effettivo del limite,
        ossia un intervallo nell'immagini della nostra funzione, come da disegno (striscia verde).
    \item Se il limite esiste effettivamente, allora possiamo trovare un'altro intervallo (questa volta
        nel dominio, ossia sull'asse x al posto che sull'asse y) tale che le \textit{immagini}
        della funzione date dalle x in questo intervallo, siano un sottoinsieme dell'intervallo
        \([L-\epsilon;L+\epsilon]\). Questo intervallo è rappresentato dalla striscia rossa nel disegno,
        e possiamo definirne la grandezza scegliendo un \(\delta > 0\).
\end{enumerate}

In maniera informale, possiamo dire che il limite esiste se per ogni striscia verde (arbitrariamente piccola),
esiste una striscia rossa tale che le immagini della striscia rossa facciano interamente parte della striscia verde.
Ciò viene precisamente espresso dall'espressione
\[
    0<|x-c|<\delta
    \implies
    |f(x)-L|<\epsilon
\]
Che significa: Quando ci troviamo nella striscia rossa (\(0<|x-c|<\delta\)),
ossia quando la distanza assoluta dal punto del limite \(c\) è minore di \(\delta\),
allora avremo necessariamente la condizione che le immagini si trovino nella striscia verde
(\(|f(x)-L|<\epsilon\)), ossia che la differenza fra i valori della funzione nei nostri punti e il
valore del limite effettivo sia minore di \(\epsilon\).

\textbf{\color{red}Importante:} non è strettamente necessaria che la funzione
\(f(x)\) sia definita al punto (c), ma solo attorno.

\newpage

\subsection{Asintoti}

\subsubsection{Definizione generale}

\sdefinition{Asintoto}{
    Un \textit{asintoto} di una curva è una retta tale che quando la funzione tende a un certo valore, finito o
    infinito, la distanza tra la retta e la funzione tende a 0
    sia nelle coordinate x e y.
}

\textbf{\color{red}Importante:} una funziona potrebbe presentare un asintoto orizzontale o obliquo
solo nella direzione positiva o negativa, o due asintoti diversi nelle direzioni positive e negative.
Gli asintoti obliqui e orizzontali non sono quindi necessariamente simmetrici.

\subsubsection{Come trovare gli asintoti orizzontali}

Bisogna verificare che il limite della funzione
all'infinito o meno infinito sia finito:\\
\[\lim_{x \to \pm\infty} f(x) = a \neq \pm\infty\]
In questo caso avremo quindi un asintoto orizzontale \(y=a\)
verso destra, sinistra o ambo le direzioni.

\subsubsection{Come trovare gli asintoti obliqui}

Per studiare gli asintoti obliqui dobbiamo studiare più in generale il concetto di equivalenza asintotica.

\sdefinition{Equivalenza sintotica}{
    Siano \(f(x)\) e \(g(x)\) due funzioni definite in un intorno di \(x_0\)
    e \(g(x)\neq 0\) se \(x \neq x_0\).
    Diciamo che \(f(x)\) e \(g(x)\) sono asintoticamente equivalenti per \(x \to x_0\)
    se il loro rapporto tende ad 1:
    \[ \lim_{x \to x_0} \frac{f(x)}{g(x)} = 1 \]
}

Per gli asintoti obliqui dobbiamo considerare l'equivalenza asintotica
per \(x \to \pm \infty\).
Senza perdita di generalità, abbiamo un asintoto obliquo per \(x \to +\infty\)
se la funzione è asintotica ad una retta \(y=mx+q\) per \(x \to +\infty\).
L'asintoto è quindi presente se
\[
    \lim_{x \to \infty} \frac{f(x)}{mx} = 1
\]
Dal momento che non conosciamo la pendenza \(m\) possiamo scrivere 
\[
    \lim_{x \to \infty} \frac{f(x)}{x} = m
\]
e controllare che \(m\) sia un numero finito.
È da notare come abbiamo tralasciato il termine \(q\) in quanto l'equivalenza asintotica ne è indipendente.

Una volta stabilistico che \(f(x)\) è asintoticamente equivalente ad una retta con pendenza \(m\),
possiamo trovare l'ordinata all'origine \(q\) calcolando la differenza fra
\(f(x)\) e \(mx+q\) con \(x \to \infty\).
\[
    q = \lim_{x\to\infty} \left(f(x)-mx\right)
\]
E per cui abbiamo stabilito la presenza di un asintoto obliquo verso destra
con funzione \(y=mx\).

\subsubsection{Come trovare gli asintoti verticali}

Per trovare gli asintoti verticali bisogna guardare il limite tendente da destra e da sinistra al punto
di interesse, e verificare che i due limiti siano \(+\infty\) o \(-\infty\) (non necessariamente simmetrici).

In tal caso abbiamo un asintoto verticale con funzione \(x=a\) dove \(a\) è il punto di interesse.

Ci sono quindi quattro tipi di geometria attorno all'asintoto a seconda del valore dei limiti.

\begin{tikzpicture}

    % 1. Asintoto verticale con la funzione che va verso ∞ su entrambi i lati
    \begin{axis}[
        width=6cm, height=6cm,
        title={$\frac{1}{x-1}$},
        domain=0.1:1.9,
        samples=100,
        ytick=\empty,
        axis lines=middle,
        xmin=0, xmax=2, ymin=-10, ymax=10,
        every axis y label/.style={at=(current axis.above origin),anchor=south},
        every axis x label/.style={at=(current axis.right of origin),anchor=west},
        ]
        \addplot [thick, red] {1/(x-1)};
        \draw [dashed] (axis cs:1,-10) -- (axis cs:1,10);
    \end{axis}
    
    % 2. Asintoto verticale con la funzione che va verso -∞ su entrambi i lati
    \begin{axis}[
        at={(6cm,0)},
        width=6cm, height=6cm,
        title={$\frac{1}{1-x}$},
        domain=0.1:1.9,
        samples=100,
        ytick=\empty,
        axis lines=middle,
        xmin=0, xmax=2, ymin=-10, ymax=10,
        every axis y label/.style={at=(current axis.above origin),anchor=south},
        every axis x label/.style={at=(current axis.right of origin),anchor=west},
        ]
        \addplot [thick, blue] {1/(1-x)};
        \draw [dashed] (axis cs:1,-10) -- (axis cs:1,10);
    \end{axis}
    
    % 3. Asintoto verticale con la funzione che va verso ∞ da un lato e verso -∞ dall'altro lato
    \begin{axis}[
        at={(0,-6cm)},
        width=6cm, height=6cm,
        title={$\frac{1}{{(x-1)}^2}$},
        domain=0.1:1.9,
        samples=100,
        ytick=\empty,
        axis lines=middle,
        xmin=0, xmax=2, ymin=-10, ymax=10,
        every axis y label/.style={at=(current axis.above origin),anchor=south},
        every axis x label/.style={at=(current axis.right of origin),anchor=west},
        ]
        \addplot [thick, green] {1/(5*x-5)/(5*x-5)};
        \draw [dashed] (axis cs:1,-10) -- (axis cs:1,10);
    \end{axis}
    
    % 4. Asintoto verticale con la funzione che si avvicina a un valore finito su un lato e va verso ∞ o -∞ sull'altro lato
    \begin{axis}[
        at={(6cm,-6cm)},
        width=6cm, height=6cm,
        title={$-\frac{1}{{(x-1)}^2}$},
        domain=0.1:1.9,
        samples=100,
        ytick=\empty,
        axis lines=middle,
        xmin=0, xmax=2, ymin=-10, ymax=10,
        every axis y label/.style={at=(current axis.above origin),anchor=south},
        every axis x label/.style={at=(current axis.right of origin),anchor=west},
        ]
        \addplot [thick, green] {-1/(5*x-5)/(5*x-5)};
        \draw [dashed] (axis cs:1,-10) -- (axis cs:1,10);
    \end{axis}
    
\end{tikzpicture}

Oltre a questi tipi di geometria, è possibile che la funzione tenda a \(\pm\infty\)
solo da una parte, come per esempio la funzione \(y=\ln(x)\).

\newpage

\subsection{La continuità di una funzione reale (esempi di discontinuità)}

\subsubsection{Definizione di funzione continua}

\sdefinition{Continuità di una funzione reale}{
    Data una funzione reale \(f(x)\), la funzione è continua nel punto \(x=c\) se
    \[
        \lim_{x\to c} = f(c)
    \]
}

In altre parole la funzione è continua nel punto se il limite esiste (sia da destra che da sinistra, coincidendo)
e la funzione è in qul punto è equivalente al valore del limite.
Una funzione è continua in un intervallo se è continua in tutti i punti dell'intervallo.

\subsubsection{Esempi di funzione continua}
\begin{enumerate}
    \item La funzione \(f(x)=x^4\) ha come \(\lim_{x \to 0} x^4=0\) e \(f(0)=0\):\\
    \textrightarrow La funzione è continua nel punto 0.
\end{enumerate}

\subsubsection{Esempi di funzione discontinua}
\begin{enumerate}
    \item La funzione \(f(x)=\frac{1}{x}\) non è continua poiche \(f(0)=\text{indeterminato}\).\\
        Essa è però continua localmente in tutti gli intervalli che non comprendono il suo
        asintoto verticale, come ad esempio l'intervallo \(x \in [1, +\infty[\).
    \item Generalmente, se si moltiplica una funzione \(f(x) \cdot \frac{x}{x}\), la funzione
        rimane la stessa ma si introduce un punto di discontinuità in 0, poiché il limite
        a 0 potrebbe esistere ma la funzione è indeterminata in quel punto indeterminato.
\end{enumerate}

\newpage
\subsection{Calcolo di limiti (somma, prodotto, ...)}

\subsubsection{Lista delle proprietà dei limiti}
Assumiamo che \(f(x)\) e \(g(x)\) siano due funzioni reali,
che \(\lim_{x \to a} f(x)\) e \(\lim_{x \to a} g(x)\) esistano dove \(c \in \mathbb{R}\) e
\(a \in \text{Dom}_f, a \in \text{Dom}_g\):
\begin{enumerate}
    \item \textbf{Linearità:}
    \begin{equation*}
        \lim_{x \to a} c f(x) = c \lim_{x \to a} f(x)
    \end{equation*}
    In altre parole, possiamo ``estrarre'' una costante moltiplicativa da un limite.
    
    \item \textbf{Somma e sottrazione:}
    \begin{equation*}
        \lim_{x \to a} \left(f(x) \pm g(x)\right) = \lim_{x \to a} f(x) \pm \lim_{x \to a} g(x)
    \end{equation*}
    Quindi, per calcolare il limite di una somma o differenza, basta prendere il limite delle
    singole parti e poi sommarle o sottrarle con il segno appropriato. Questo fatto vale per
    qualsiasi numero di funzioni separate da ``+'' o ``-''.
    
    \item \textbf{Moltiplicazione:}
    \begin{equation*}
        \lim_{x \to a} \left(f(x) g(x)\right) = \lim_{x \to a} f(x) \cdot \lim_{x \to a} g(x)
    \end{equation*}
    Calcoliamo i limiti dei prodotti allo stesso modo in cui calcoliamo i limiti delle somme
    o differenze. Prendiamo semplicemente il limite delle singole parti e poi le
    moltiplichiamo. Anche in questo caso, questo fatto non è limitato a solo due funzioni.
    
    \item \textbf{Divisione:}
    \begin{equation*}
        \lim_{x \to a} \left( \frac{f(x)}{g(x)} \right) = \frac{\lim_{x \to a} f(x)}{\lim_{x \to a} g(x)}, \quad \text{purché } \lim_{x \to a} g(x) \ne 0
    \end{equation*}
    Come indicato nella frase, dobbiamo preoccuparci solo del limite nel denominatore che
    tende a zero quando calcoliamo il limite di un quoziente. Se fosse zero, avremmo un errore
    di divisione per zero che dobbiamo evitare.
    
    \item \textbf{Potenza:}
    \begin{equation*}
        \lim_{x \to a} [f(x)]^n = \left[\lim_{x \to a} f(x)\right]^n, \quad \text{dove } n \text{ è un numero reale}
    \end{equation*}
    In questa proprietà, \(n\) può essere qualsiasi numero reale (positivo, negativo, intero,
    frazione, irrazionale, zero, \dots). Nel caso in cui \(n\) sia un numero intero, questa
    regola può essere vista come un'estensione del caso 3.
    
    \item \textbf{Radice:}
    \begin{equation*}
        \lim_{x \to a} \left[ \sqrt[n]{f(x)} \right] = \sqrt[n]{\lim_{x \to a} f(x)}
    \end{equation*}
    Questo è solo un caso speciale dell'esempio precedente.
    
    \item \textbf{Costante:}
    \begin{equation*}
        \lim_{x \to a} c = c, \quad \text{c è un numero reale qualsiasi}
    \end{equation*}
    Il limite di una costante è semplicemente la costante stessa.
    
    \item \textbf{Identità:}
    \begin{equation*}
        \lim_{x \to a} x = a
    \end{equation*}
    Questa proprietà afferma che il limite dell'identità \(f(x) = x\) quando \(x\)
    tende ad \(a\) è semplicemente \(a\). Questo è intuitivo, poiché man mano che \(x\) si
    avvicina ad \(a\), la funzione \(f(x) = x\) si avvicina direttamente al valore \(a\).
    
    \item \textbf{Potenza della variabile:}
    \begin{equation*}
        \lim_{x \to a} x^n = a^n
    \end{equation*}
    Questo è un caso speciale della proprietà 5 usando l'identità \(f(x) = x\).
\end{enumerate}

\newpage
\section{Derivate}
\subsection{Definizione di derivata e applicazione in casi concreti}

\subsubsection{Definzione}

\sdefinition{Derivata}{
    La \textit{derivata} di una funzione reale \(f(x)\) è definita come
    \[
        f'(x) = \lim_{\Delta x \to 0} \frac{f(x + \Delta x) - f(x)}{\Delta x}
    \]
}

La derivata può essere scritta in diversi modi.

\begin{table}[h]
    \centering
    \renewcommand{\arraystretch}{1.5}
    \begin{tabular}{|c|c|c|c|}
    \hline
    \textbf{Descrizione} & \textbf{Prima derivata} & \textbf{Seconda derivata} & \textbf{n-esima derivata} \\
    \hline
    Notazione di Lagrange & $f'(x)$ & $f''(x)$ & $f^{(n)}(x)$ \\
    \hline
    Notazione di Leibniz & $\frac{d}{dx} f(x)$ & $\frac{d^2}{dx^2} f(x)$ & $\frac{d^n}{dx^n} f(x)$ \\
    \hline
    Notazione di Newton & $\dot{f}$ & $\ddot{f}$ & $f^{(n)}$ \\
    \hline
    \end{tabular}
    \caption{Notazioni comuni per le derivate}
    \label{tab:derivate}
\end{table}

Nota che la notazione di Newton non specifica quale variabile stiamo derivando. Solitamente
viene utilizzata in fisica per derivare rispettivamente alla variabile temporale \(t\).
Se abbiamo una funzione \(x^3+a\), potremmo direttamente scrivere \((x^3+a)^{'}\)
per rappresentarne la prima derivata.

\subsection{Significato geometrico della derivata}
















    \begin{minipage}{0.5\textwidth}
        \begin{tikzpicture}[
            scale=2,
            declare function={
                func(\x) = \x*sin(\x r);
                Width=3;
                Height=2;
                Ax=0.75;
                Bx=2.25;
                SlopeMargin=0.25;
                M=(func(Bx) - func(Ax)) / (Bx - Ax);
                Q=func(Ax) - M * Ax;
                slopeFunc(\x)=\x * M + Q;
            }
        ]
            \draw[domain=-0.5:3, smooth, variable=\x, blue, very thick] plot ({\x}, {func(\x)});
            
            \draw[->] (0, -0.25) -- (0, Height) node[right] {\(y\)};
            \draw[->] (-0.25, 0) -- (Width, 0) node[above] {\(x\)};
    
            \draw[-] (Ax, {func(Ax)}) -- node[below] {\(\Delta x\)} (Bx, {func(Ax)});
            \draw[-] (Bx, {func(Ax)}) -- node[right] {\(\Delta y\)} (Bx, {func(Bx)});
            
            \filldraw [red, thick] ({Ax - SlopeMargin}, {slopeFunc(Ax - SlopeMargin)}) -- ({Bx + SlopeMargin}, {slopeFunc(Bx + SlopeMargin)});
            
            \filldraw (Ax,{func(Ax)}) circle (1pt) node[above left] {\(A\)};
            \filldraw (Bx,{func(Bx)}) circle (1pt) node[above] {\(B\)};
        \end{tikzpicture}
    \end{minipage}
    \begin{minipage}{0.5\textwidth}
        La secante di una funzione \(f(x)\) fra un punto \(A\) e \(B\) è data da
        \[
            \frac{\Delta y}{\Delta x} = \frac{f(B)-f(A)}{B-A}
        \]
        Più rendiamo \(A\) e \(B\) vicini fra loro, più \(\Delta x\) diminuisce.
        più il \(\Delta x\)diminuisce, la pendenza della secante è sempre più rappresentativa
        del tasso di cambiamento di \(f\) nell'intervallo \([A;B]\). \\
    \end{minipage}
    
    \begin{minipage}{0.5\textwidth}
        Quando il \(\Delta x\) della pendenza è infinitesimamente piccolo, abbiamo la pendenza precisa in un punto (istantanea).
        Questa pendenza è rappresentata dalla retta tangente, che è parallela al punto sulla funzione.
        \[
            \lim_{\Delta x \to 0} \frac{\Delta y}{\Delta x}
        \]
    \end{minipage}
    \begin{minipage}{0.5\textwidth}
        \begin{tikzpicture}[
            scale=2,
            declare function={
                func(\x) = (\x+0.6)*sin((\x+0.6) r)-0.5;
                Width=3;
                Height=2;
                Ax=1.5;
                slope = -0.19697;
            }
        ]
            \draw[domain=-0.5:2.5, smooth, variable=\x, blue, very thick] plot ({\x}, {func(\x)});
            
            \draw[->] (0, -0.25) -- (0, Height) node[right] {\(y\)};
            \draw[->] (-0.25, 0) -- (Width, 0) node[above] {\(x\)};
    
            \draw[domain=0.5:2.5, smooth, variable=\x, red, very thick] plot ({\x}, {slope * \x + func(Ax) - slope * Ax});
            
            \filldraw (Ax,{func(Ax)}) circle (1pt) node[above] {\(A\)};
        \end{tikzpicture}
    \end{minipage}
    
    \begin{minipage}{0.5\textwidth}
    \begin{tikzpicture}[
        scale=2,
        declare function={
            func(\x) = \x*sin(\x r);
            Width=3;
            Height=2;
            Ax=0.75;
            Bx=1.8;
            SlopeMargin=0.25;
            M=(func(Bx) - func(Ax)) / (Bx - Ax);
            Q=func(Ax) - M * Ax;
            slopeFunc(\x)=\x * M + Q;
        }
    ]
        \draw[domain=-0.5:3, smooth, variable=\x, blue, very thick] plot ({\x}, {func(\x)});
        
        \draw[->] (0, -0.25) -- (0, Height) node[right] {\(y\)};
        \draw[->] (-0.25, 0) -- (Width, 0) node[above] {\(x\)};
    
        \draw[-] (Ax, {func(Ax)}) -- node[below] {\(\Delta x\)} (Bx, {func(Ax)});
        \draw[-] (Bx, {func(Ax)}) -- node[right] {\(\Delta y\)} (Bx, {func(Bx)});
        
        \filldraw [red, thick] ({Ax - SlopeMargin}, {slopeFunc(Ax - SlopeMargin)}) -- ({Bx + SlopeMargin}, {slopeFunc(Bx + SlopeMargin)});
    
        \filldraw (Ax,{func(Ax)}) circle (1pt) node[above left] {\(a\)};
\end{tikzpicture}
\end{minipage}
\begin{minipage}{0.5\textwidth}
    La derivata di una funzione \(f(x)\) è quindi un'altra funzione, \(f'(x)\), che
    rappresenta il cambiamento di \(f(x)\) in ogni punto. In altre parole, \(f'(x)\)
    rappresenta la pendenza della tangente in ogni \(x\) di \(f(x)\).
    Questo è precisamente rappresentato dalla definizione di derivata, ossia la pendenza
    \(\frac{\Delta y}{\Delta x}\) calcolata con il limite di \(\Delta x \to 0\).
\end{minipage}

\subsubsection{Esempio applicazione}

In fisica, la funzione della velocità \(v(t)\) rappresenta in ogni punto il cambiamento istantaneo
della posizione \(x(t)\), e l'accellerazione \(a(t)\) rappresenta in ogni punto il cambiamento
istantanea della velocità \(v(t)\).
Di conseguenza abbiamo quindi che \(x'(t)=v(t)\) e \(v'(t)=a(t)\).
Nota che quando facciamo la derivata della posizione o della velocità, l'informazione
della posizione iniziale e velocità iniziale vengono persi, in quanto le pendenze
sono indipendenti da quanto la funzione è spostata in alto o in basso.

\subsubsection{Procedimento per calcolare la tangente}

Data una funzione \(f(x)\), vogliamo trovare l'equazione della retta tangente al punto \(a\).
Assumiamo quindi che \(f'(a)\) non sia indeterminato:
\begin{enumerate}
    \item Calcolare \(f'(a)\) e trovare la pendenza (m) della tangente in quel punto;
    \item Usare la pendenza per (m) per scrivere l'equazione della tangente
        \(y=mx+q\) con la pendenza uguale a quella della tangente;
    \item Spostare la funzione della tangente in orizzontale verso il punto (a)
        moltiplicando la pendenza (m) = \(f'(a)\) al punto (a) = \(x-a\): 
        \begin{equation*}
            y=f'(a)\cdot (x-a)
        \end{equation*}
    \item Spostare la funzione della tangente in verticale verso il punto (a)
        aggiungendo la funzione originale\\ (q) = \(f(a)\): 
        \begin{equation*}
            y=f'(a)(x-a)+f(a)
        \end{equation*}
\end{enumerate}

\newpage

\subsection{Regole di derivazione}

\subsubsection{Lista delle proprietà delle derivate}

Assumiamo che \(f(x)\) e \(g(x)\) siano due funzioni derivabili e \(c\in\mathbb{R}\)
\begin{enumerate}
    \item \textbf{Derivata di una costante:}
    \begin{equation*}
        \frac{d}{dx} c = 0
    \end{equation*}
    La derivata di una costante è sempre zero, in quanto la pendenza di \(y=c\) è sempre zero.
    
    \item \textbf{Linearità:}
    \begin{equation*}
        \frac{d}{dx} cf(x) = c \frac{d}{dx} f(x)
    \end{equation*}
    La derivata di una funzione moltiplicata per una costante è uguale alla costante moltiplicata per la derivata della funzione.
    
    \item \textbf{Somma e sottrazione:}
    \begin{equation*}
        \frac{d}{dx} \left(f(x) \pm g(x)\right) = \frac{d}{dx} f(x) \pm \frac{d}{dx} g(x)
    \end{equation*}
    La derivata di una somma o sottrazione di funzioni è uguale alla somma o sottrazione delle derivate delle singole funzioni.
    
    \item \textbf{Moltiplicazione:}
    \begin{equation*}
        \frac{d}{dx} \left(f(x) g(x)\right) = f(x) \frac{d}{dx} g(x) + g(x) \frac{d}{dx} f(x)
    \end{equation*}
    La derivata del prodotto di due funzioni è uguale alla derivata del primo termine per il secondo termine più la derivata del secondo termine per il primo termine.
    
    \item \textbf{Divisione:}
    \begin{equation*}
        \frac{d}{dx} \left( \frac{f(x)}{g(x)} \right) = \frac{g(x) \frac{d}{dx} f(x) - f(x) \frac{d}{dx} g(x)}{[g(x)]^2}
    \end{equation*}
    La derivata del quoziente di due funzioni è data dalla formula del quoziente, che coinvolge le derivate del numeratore e del denominatore.
    
    \item \textbf{Potenza:}
    \begin{equation*}
        \frac{d}{dx} x^n = n x^{n-1}
    \end{equation*}
    La derivata di una funzione nella forma data ha l'esponente ridotto di un fattore \(1\), e guadagna
    un coefficiente pari all'esponente iniziale.
    
    \item \textbf{Radice:}
    \begin{equation*}
        \frac{d}{dx} \left[ \sqrt[n]{f(x)} \right] = \frac{1}{n} \left[ f(x) \right]^{\frac{1}{n} - 1} \frac{d}{dx} f(x)
    \end{equation*}
    La derivata di una radice ennesima di una funzione è data dalla formula che coinvolge la derivata della funzione stessa.
    
    \item \textbf{Funzione composta (Regola della catena):}
    \begin{equation*}
        \frac{d}{dx} f(g(x)) = f'(g(x)) \cdot g'(x)
    \end{equation*}
    La derivata di una funzione composta è il prodotto della derivata della funzione esterna valutata nella funzione interna, per la derivata della funzione interna.
        
    \item \textbf{Derivata della funzione esponenziale:}
    \begin{equation*}
        \frac{d}{dx} [e^{f(x)}] = e^{f(x)} \frac{d}{dx} f(x)
    \end{equation*}
    La derivata della funzione esponenziale è la funzione esponenziale stessa moltiplicata per la derivata dell'esponente.
    
    \item \textbf{Derivata del logaritmo:}
    \begin{equation*}
        \frac{d}{dx} [\ln f(x)] = \frac{f'(x)}{f(x)}
    \end{equation*}
    La derivata del logaritmo naturale di una funzione è uguale alla derivata della funzione divisa per la funzione stessa.
\end{enumerate}

\newpage

\subsection{Teorema di Bernoulli-de l'Hôpital}

\subsubsection{Le 7 forme indeterminate}

Le sette forme indeterminate sono
\begin{center}
    \[
        \frac{0}{0}, \quad
        \frac{\infty}{\infty}, \quad
        0 \cdot \infty, \quad
        \infty - \infty, \quad
        0^0, \quad
        \infty^0, \quad
        1^\infty
    \]
\end{center}

\subsubsection{Definizione del teorema}

\stheorem{Teorema di Bernoulli-de l'Hôpital}{
    Consideriamo due funzioni reali \(f(x)\) e \(g(x)\) che sono differenziabili
    in un intorno di \(x_0\in\mathbb{R}\) (non necessariamente in \(c\)).
    Senza perdita di generalità, se abbiamo un limite
    \[ \lim_{x \to x_0} \frac{f(x)}{g(x)} \]
    dove \(\frac{f(x_0)}{g(x_0)}\) risulta in una forma indeterminata, allora
    \[ \lim_{x \to x_0} \frac{f(x)}{g(x)} = \lim_{x \to x_0} \frac{f'(x)}{g'(x)} \]
}

\subsubsection{Esempio}
\begin{equation*}
    \lim_{x \to +\infty} \frac{4x}{2x} = \frac{\infty}{\infty}
    \overset{\text{BH}}{\longrightarrow} \lim_{x \to +\infty} \frac{4}{2} = 2
\end{equation*}

\newpage
\subsection{Studio della monotonia e ricerca degli estremi di una funzione}
\subsubsection{Definizione monotonia}
PAOLO PORCODIO SCRIVI TU PF NN SO COSA sia

\subsubsection{CACCA CULO?? NON SO CHE TITOLO METTERe}
Poiché la derivata \(f'(x)\) rappresenta il tasso di variazione di \(f(x)\), assumendo che
\(f(a)\) sia definita.

\begin{itemize}
    \item Se \(f'(a) >; 0\), allora \(f(x)\) è crescente in \(x = a\)
    \item Se \(f'(a) <; 0\), allora \(f(x)\) è decrescente in \(x = a\)
    \item Se \(f'(a) = 0\), allora \(f(x)\) è un punto critico in \(x = a\)
    \item Se \(f'(a)\) non è definita, allora \(f(x)\) è un punto critico in \(x = a\)
    (punto angoloso)
\end{itemize}


\newpage
\subsection{Studio della concavità e ricerca dei punti di flesso}
\subsubsection{Definizione di concavità}
La concavità di una funzione \(f(x)\) descrive la direzione della sua curvatura, che può
essere verso l'alto quando \(f''(x)>0\) o verso il basso quando \(f''(x)<0\).\\
Inoltre la concavità può essere crescente quando nel pressi della concavità \(f'(x)>0\) e 
decrescente quando \(f'(x)<0\).

\subsubsection{Definizione di punto di flesso}
Il punto di flesso è il punto in cui la concavità della funzione cambia, ossia dove la
funzione passa da concava verso l'alto a concava verso il basso o viceversa:
\begin{equation*}
    f''(c) = 0 \text{ e } f''(x) \text{ cambia segno in } x=c
\end{equation*}

\subsubsection{In più}
Tutti i punti di flesso di una funzione \(f(x)\) sono determinati dove la seconda derivata
della funzione si annulla \(f''(x)=0\), ma non tutte le seconde derivate che hanno valore 0
\(f''(x)=0\) sono dei punti di flesso.

\newpage
\section{Integrali}
\subsection{Integrale definito e sue applicazioni}
\subsubsection{Definizione}
Con integrale definito si può considerare quello di Riemann, il quale può essere definito
con la somma infinitesimale di rettangoli con base tendente a 0 e l'altezza pari a un punto
all'interno della base:
\begin{wrapfigure}{l}{10cm}
    \begin{center}
        \begin{tikzpicture}[
            scale=1.25,
            declare function={
                func(\x) = 0.1 * (\x - 2) * (\x - 2) + 2;
                Width=7.5;
                Height=5;
                A = 1;
                B = 7;
                N = 7;
                Delta = {(B-A) / N};
            }
        ]
            \draw[->] (0, 0) -- (0, Height) node[right] {\(y\)};
            \draw[->] (0, 0) -- (Width, 0) node[above] {\(x\)};
    
            \pgfmathtruncatemacro\END{N-1}
            \foreach \x in {0,1,...,\END} {
                % rectangle
                \fill [green, opacity=0.2]
                    ({A + Delta * \x}, 0) rectangle ({A + Delta * (\x+1)}, {func(A + Delta * (\x + 0.5))});
                
                \draw[-] ({A + Delta * \x}, 0)
                    -- ({A + Delta * \x}, {func(A + Delta * (\x + 0.5))});

                \draw[-, dashed] ({A + Delta * (\x + 0.5)}, 0)
                    node[below] {\(x_{\x}^*\)}
                    -- ({A + Delta * (\x + 0.5)}, {func(A + Delta * (\x + 0.5))});

                % line over rectangle
                \draw[-] ({A + Delta * \x}, {func(A + Delta * (\x + 0.5))})
                    -- ({A + Delta * (\x+1)}, {func(A + Delta * (\x + 0.5))});
            }
    
            \draw[-, dashed, red, ultra thick] (A, 0) -- (A, {func(A)}) node[above, black] {\(a\)};
            \draw[-, dashed, red, ultra thick] (B, 0) -- (B, {func(B)}) node[right, black] {\(b\)};
    
            \draw[domain=0:Width, smooth, variable=\x, blue, ultra thick] plot ({\x}, {func(\x)});
        \end{tikzpicture}
    \end{center}
\end{wrapfigure}
\wrapfill
\vspace*{-1.25cm}

{\color{red}{\textbf{Importante:}}} esistono funzioni che non sono integrabili nel senso di
Riemann, ad esempio una funzione:
\[
    f(x)=
    \begin{cases}
    1 & x \in \mathbb{Q} \\
    0 & x \notin \mathbb{Q}
    \end{cases}
\]
Questa funzione ha un integrale, ma non si può calcolare con la somma di Riemann.

\subsubsection{Keep yapping}
L'integrale di una funzione non strettamente positiva può essere negativo, infatti i pezzi
sotto l'asse delle x rappresentano aree con valori negativi, quindi l'integrale rappresenta
l'area segnata sotto la curva.

\subsubsection{Le sue applicazioni}
L'integrale di un intervallo di una funzione probabilistica rappresenta la probabilità di
ritrovarsi in suddetto intervallo.

\newpage
\subsection{Calcolo di aree di superfici racchiuse tra i grafici di due funzioni}
Creare una funzione che rappresenti la distanza tra due funzioni \(f(x) e g(x)\) e integrarla.

\subsubsection{Come calcolare l'area}
\begin{enumerate}
    \item Calcolare i limiti di integrazione, come ad esempio i punti di intersezioni tra le
        due funzioni;
    \item Integrare la differenza delle due funzioni (la più piccola sottratta alla più grande)
        per avere una superficie positiva ({\color{red}\underline{Occhio al segno!}})
\end{enumerate}\phantom{}

\begin{wrapfigure}{l}{7.5cm}
    \begin{tikzpicture}[x=4cm, y=4cm, >=Stealth, declare function={
        func1(\x) = sin(3.14*\x/2 r);
        func2(\x) = \x*\x;
        a=0.25;
        b=0.9;
    }]
        % area under func1 
        \fill[cyan!30!white] plot[domain=a:b] (\x,{func1(\x)}) -- plot[domain=b:a] (\x,{0}) -- cycle;
        

        % area under func2
        \fill[white] plot[domain=a:b] (\x,{func2(\x)}) -- plot[domain=b:a] (\x,{0}) -- cycle;
        
        % func1
        \draw[blue, -, thick] plot[domain=-0.3:1.25] (\x,{func1(\x)});
        
        % func2
        \draw[red, -, thick]  plot[domain=-0.5:1.2] (\x,{func2(\x)});

        \draw[-, dashed] (a, 0) node[below] {\(a\)} -- (a, {func1(a)});
        \draw[-, dashed] (b, 0) node[below] {\(b\)} -- (b, {func1(b)});

        \draw[-] (-.5,0) -- (1.25, 0)
            node[below left=4pt] {\(x\)};
        \draw[-] (0,-.5) -- (0,1.5)
            node[below left=4pt] {\(y\)};
    \end{tikzpicture} 
\end{wrapfigure}

Data una funzione \(y=f(x)\) e \(y=g(x)\), l'area racchiusa dalle due funzioni
nell'intervallo \(I=[a;b]\) è data da
\[
    A=\integral[a][b][f(x)-g(x)][x]
\]
assumendo che \(f(x)\geq g(x)\) quando \(x\in I\). \\
Nota che \(A\geq 0\). \\
Se \(f(x)<g(x)\) per qualche \(x\in I\), questa formula non funzionerà. Tuttavia, è comunque possibile dividere l'integrale in più integrali in ogni punto in cui \(f(x) - g(x)\) cambia segno.
Per rimuovere il vincolo del segno potremmo dire
\[
    A = \integral[a][b][|f(x)-g(x)|][x]
\]
\wrapfill

\subsubsection{Esempio}
Calcolo dell'area tra le funzioni \(f(x)=x^2\) e \(g(x)=x\)
\begin{enumerate}
    \item Trovare i punti di intersezione:
        \begin{equation*}
            x^2=x \Rightarrow x^2-x=0 \Rightarrow x(x-1)=0 \longrightarrow
            x_1=0, x_{2,3}=1
        \end{equation*}
    \item Integrare la differenza delle due funzioni (la più grande -- la più piccola):
        \begin{align*}
            &A=\integral[0][1][x-x^2][x]\\
            &A=\integral[0][1][x][x] - \integral[0][1][x^2][x]\\
            &A=\left[\frac{x^2}{2} - \frac{x^3}{3}\right]_0^1\\
            &A=(\frac{1^2}{2}-\frac{0^2}{2})-(\frac{1^3}{3}-\frac{0^3}{3})\\
            &A=\frac{1}{2}-\frac{1}{3}=\frac{1}{6}
        \end{align*}
\end{enumerate}



\newpage
\subsection{Integrale indefinito, primitve di una funzione}
\subsubsection{Definizione di integrale indefinito}
L'operatore per trovare una funzione primitiva è chiamato \textbf{integrale indefinito}
    \[
        \integral[f(x)][x]=F(x)+C,
        \quad C\in\mathbb{R}
    \]
La funzione da integrare (integranda) è delimitata dal simbolo dell'integrale \(\int\)
e un differenziale della variabile di integrazione \(dx\).

Una funzione ha infiniti primitivi, quindi il termine \(+ C\). Questo significa essenzialmente
che la derivata di una funzione è la stessa quando la funzione è traslata verso l'alto o verso
il basso, il tasso di cambiamento è lo stesso. Invertendo il processo non conosciamo la
traslazione verso l'alto o verso il basso della funzione originale.
\[
    f(x)=\integral[\frac{df}{dx}][x] + C
\]
per qualche specifico \(C\).

\subsubsection{Definizione di primitiva (o antiderivata)}
La primitiva di una funzione è una funzione tale che la sua derivata è uguale alla funzione
originale, ossia:
\[
    F'(x)=f(x)    
\]

Trovando una qualsiasi primitiva \(F(x)\), si può trovare una classe di infinite primitive
aggiungendo una qualsiasi costante:
\[
    F(x)+C,
    \quad C\in\mathbb{R}  
\]

\subsubsection{Esempio concreto}
La derivata della posizione è la velocità, ma l'informazione della posizione iniziale \(x_0\)
viene persa. Facendo l'integrale della velocità per ritrovare \(x(t)\), la costante \(C\)
rappresenta la posizione iniziale \(x_0\)

\newpage
\part{Geometria}

\section{Geometria analitica nel piano}

\subsection{Equazione della retta, rette parallele e rette perpendicolari}

\subsubsection{Equazione della retta}
L'equazione di una retta nel piano cartesiano è generalmente espressa nella forma:
\[
    y = mx + q
\]
dove \(m\) è il coefficiente angolare che determina la pendenza della retta, e \(q\) è
l'intercetta con l'asse \(y\), che indica dove la retta attraversa l'asse \(y\).
Il coefficiente angolare \(m\) è calcolato come il rapporto tra la variazione verticale
(\(\Delta y\)) e la variazione orizzontale (\(\Delta x\)) tra due punti qualsiasi sulla retta:
\[
    m = \frac{\Delta y}{\Delta x}
\]

\begin{tikzpicture}
    \draw[thick, ->] (-3,0) -- (3,0) node[right] {\(x\)};
    \draw[thick, ->] (0,-3) -- (0,3) node[above] {\(y\)};
    \draw[domain=-2.5:2.5, smooth, variable=\x, blue] plot ({\x}, {0.5*\x + 1});
    \node at (2.6, 2.6) {\(y = 0.5x + 1\)};
\end{tikzpicture}

\newpage
\subsubsection{Rette parallele}
Due rette sono parallele se e solo se i loro coefficienti angolari sono uguali:
\[
    m_1 = m_2
\]
Questo implica che le rette hanno la stessa pendenza e quindi non si intersecano mai.\\
Ad esempio, le rette di equazioni \(y = 2x + 3\) e \(y = 2x - 1\) sono parallele perché
entrambe hanno un coefficiente angolare di 2.

\begin{tikzpicture}
    \draw[thick, ->] (-3,0) -- (3,0) node[right] {\(x\)};
    \draw[thick, ->] (0,-3) -- (0,3) node[above] {\(y\)};
    \draw[domain=-1.5:0.9, smooth, variable=\x, blue] plot ({\x}, {2*\x + 1});
    \draw[domain=-0.7:1.7, smooth, variable=\x, red] plot ({\x}, {2*\x - 1});
    \node[color=blue] at (1.7, 3.1) {\(y = 2x + 1\)};
    \node[color=red] at (2.7, 2.45) {\(y = 2x - 1\)};
\end{tikzpicture}

\newpage
\subsubsection{Rette perpendicolari}
Due rette sono perpendicolari se e solo se il prodotto dei loro coefficienti angolari è \(-1\):
\[
    m_1 \cdot m_2 = -1
\]
Questo significa che le rette formano un angolo retto (90 gradi) l'una con l'altra. Ad esempio,
le rette di equazioni \(y = -\frac{1}{2}x + 1\) e \(y = 2x + 1\) sono perpendicolari perché il
prodotto dei loro coefficienti angolari \(-\frac{1}{2}\) e \(2\) è \(-1\).

\begin{tikzpicture}
    \draw[thick, ->] (-3,0) -- (3,0) node[right] {$x$};
    \draw[thick, ->] (0,-3) -- (0,3) node[above] {$y$};
    \draw[domain=-2.5:2.5, smooth, variable=\x, blue] plot ({\x}, {-0.5*\x + 1});
    \draw[domain=-1.5:1, smooth, variable=\x, red] plot ({\x}, {2*\x + 1});
    \node at (1.8, -0.5) {$y = -0.5x + 1$};
    \node at (2, 2.7) {$y = 2x + 1$};
\end{tikzpicture}

\newpage
\subsection{Definizione e equazione della parabola}

\subsubsection{Definizione di parabola}
Una parabola è il luogo dei punti equidistanti da un punto fisso, detto fuoco, e una retta
fissa, detta direttrice. Il fuoco e la direttrice determinano l'orientamento e l'apertura
della parabola. La parabola ha un asse di simmetria che passa per il fuoco e perpendicolare
alla direttrice.

\subsubsection{Equazione della parabola}
L'equazione di una parabola con vertice nell'origine e asse di simmetria parallelo all'asse
\(y\) è:
\[
    y = ax^2
\]
dove \(a\) è un parametro che determina l'apertura della parabola. Se \(a > 0\), la parabola
è aperta verso l'alto; se \(a < 0\), la parabola è aperta verso il basso.

\begin{tikzpicture}
    \draw[thick, ->] (-3,0) -- (3,0) node[right] {\(x\)};
    \draw[thick, ->] (0,-3) -- (0,3) node[above] {\(y\)};
    \draw[domain=-1.7:1.7, smooth, variable=\x, blue] plot ({\x}, {\x*\x});
    \node at (2.5, 3) {\(y = x^2\)};
\end{tikzpicture}

\newpage
\subsection{Definizione e equazione della circonferenza, rette tangenti}

\subsubsection{Definizione di circonferenza}
La circonferenza è il luogo dei punti equidistanti da un punto fisso detto centro. La distanza
costante è il raggio della circonferenza. Il centro e il raggio determinano la posizione e la
dimensione della circonferenza.

\subsubsection{Equazione della circonferenza}
L'equazione della circonferenza con centro nell'origine e raggio \(r\) è:
\[
    x^2 + y^2 = r^2
\]
Se il centro della circonferenza è il punto \((h, k)\), l'equazione generale della
circonferenza diventa:
\[
    (x - h)^2 + (y - k)^2 = r^2
\]

\begin{tikzpicture}
    \draw[thick, ->] (-3,0) -- (3,0) node[right] {\(x\)};
    \draw[thick, ->] (0,-3) -- (0,3) node[above] {\(y\)};
    \draw (0,0) circle (2);
    \node at (2.5, 1.7) {\(x^2 + y^2 = 4\)};
\end{tikzpicture}

\subsubsection{Rette tangenti}
Una retta è tangente a una circonferenza se e solo se la distanza dal centro della
circonferenza alla retta è uguale al raggio della circonferenza. La tangente tocca la
circonferenza in un solo punto.

\begin{tikzpicture}
    \draw[thick, ->] (-3,0) -- (3,0) node[right] {\(x\)};
    \draw[thick, ->] (0,-3) -- (0,3) node[above] {\(y\)};
    \draw (0,0) circle (2);
    \draw[domain=-2.5:2.5, smooth, variable=\x, red] plot ({\x}, {2});
    \node at (2, 2.3) {\(y = 2\)};
\end{tikzpicture}

\newpage
\subsection{Problemi di intersezione}

\subsubsection{Intersezione tra rette}
Per trovare i punti di intersezione tra due rette, si risolvono le loro equazioni
simultaneamente. Ad esempio, per trovare i punti di intersezione tra due rette di equazioni
\(y = m_1x + q_1\) e \(y = m_2x + q_2\), si risolve il sistema:
\[
    \begin{cases}
    y = m_1x + q_1 \\
    y = m_2x + q_2
    \end{cases}
\]
Risolvendo il sistema si ottiene il punto di intersezione \((x_0, y_0)\), dove:
\[
    x_0 = \frac{q_2 - q_1}{m_1 - m_2}, \quad y_0 = m_1x_0 + q_1
\]

\begin{tikzpicture}
    \draw[thick, ->] (-3,0) -- (3,0) node[right] {\(x\)};
    \draw[thick, ->] (0,-3) -- (0,3) node[above] {\(y\)};
    \draw[domain=-2.5:1.9, smooth, variable=\x, blue] plot ({\x}, {\x + 1});
    \draw[domain=-3:2.5, smooth, variable=\x, red] plot ({\x}, {-0.5*\x + 1});
    \node[color=blue] at (2.5, 2.5) {\(y = x + 1\)};
    \node[color=red] at (-2, 2.8) {\(y = -0.5x + 1\)};
\end{tikzpicture}

\subsubsection{Intersezione tra retta e circonferenza}
Per trovare i punti di intersezione tra una retta e una circonferenza, si risolvono
simultaneamente le loro equazioni. Ad esempio, per una retta \(y = mx + q\) e una
circonferenza \(x^2 + y^2 = r^2\), si risolve il sistema:
\[
    \begin{cases}
    y = mx + q \\
    x^2 + y^2 = r^2
    \end{cases}
\]
Sostituendo l'equazione della retta nell'equazione della circonferenza, si ottiene
un'equazione quadratica in \(x\):
\[
    x^2 + (mx + q)^2 = r^2
\]
Risolvendo questa equazione, si trovano le coordinate dei punti di intersezione.

\begin{tikzpicture}
    \draw[thick, ->] (-3,0) -- (3,0) node[right] {\(x\)};
    \draw[thick, ->] (0,-3) -- (0,3) node[above] {\(y\)};
    \draw (0,0) circle (2);
    \draw[domain=-2.5:1.75, smooth, variable=\x, red] plot ({\x}, {\x + 1});
    \node[color=red] at (2, 2) {\(y = x + 1\)};
\end{tikzpicture}

\newpage
\subsection{Problemi di distanza}

\subsubsection{Distanza tra due punti}
La distanza tra due punti \(A(x_1, y_1)\) e \(B(x_2, y_2)\) nel piano cartesiano è data dalla
formula:
\[
    d = \sqrt{(x_2 - x_1)^2 + (y_2 - y_1)^2}
\]
Questa formula deriva dal teorema di Pitagora applicato al triangolo rettangolo formato dai
punti \(A\) e \(B\) e le loro proiezioni sugli assi.

\begin{tikzpicture}
    \draw[thick, ->] (-3,0) -- (3,0) node[right] {\(x\)};
    \draw[thick, ->] (0,-3) -- (0,3) node[above] {\(y\)};
    \filldraw[blue] (1,2) circle (2pt) node[anchor=south] {\(A(x_1, y_1)\)};
    \filldraw[red] (2,1) circle (2pt) node[anchor=west] {\(B(x_2, y_2)\)};
    \draw[dashed] (1,2) -- (2,2) -- (2,1);
    \draw (1,2) -- (2,1);
\end{tikzpicture}

\subsubsection{Distanza punto-retta}
La distanza di un punto \(P(x_0, y_0)\) da una retta di equazione \(Ax + By + C = 0\) è data
da:
\[
    d = \frac{|Ax_0 + By_0 + C|}{\sqrt{A^2 + B^2}}
\]
Questa formula può essere derivata considerando la perpendicolare dalla retta al punto \(P\)
e utilizzando le proprietà geometriche dei triangoli.

\begin{tikzpicture}
    \draw[thick, ->] (-3,0) -- (3,0) node[right] {\(x\)};
    \draw[thick, ->] (0,-3) -- (0,3) node[above] {\(y\)};
    \draw[domain=-2.5:2.5, smooth, variable=\x, red] plot ({\x}, {0.5*\x + 1});
    \filldraw[blue] (0.8,2.4) circle (1.5pt) node[anchor=south] {\(P(x_0, y_0)\)};
    \draw[dashed] (0.8,2.4) -- node[anchor=east] {d} (1.2,1.6);
    \filldraw (1.2,1.6) circle (1.5pt);
\end{tikzpicture}

\newpage
\section{Geometria vettoriale}
\subsection{Vettori linearmente indipendenti (collinearità e complanarità)}

\subsubsection{Vettori linearmente indipendenti}
Due o più vettori sono linearmente indipendenti se nessuno di essi può essere scritto come
combinazione lineare degli altri. In termini matematici, i vettori
\(\vec{v}_1, \vec{v}_2, \ldots, \vec{v}_n\) sono linearmente indipendenti se l'equazione:
\[
    c_1 \vec{v}_1 + c_2 \vec{v}_2 + \cdots + c_n \vec{v}_n = \vec{0}
\]
implica che \(c_1 = c_2 = \cdots = c_n = 0\).

\subsubsection{Collinearità}
Due vettori \(\vec{a}\) e \(\vec{b}\) sono collineari se esiste un numero reale \(\lambda\)
tale che:
\[
    \vec{a} = \lambda \vec{b}
\]

\subsubsection{Complanarità}
Tre vettori \(\vec{a}, \vec{b}, \vec{c}\) sono complanari se esistono due numeri reali
\(\lambda\) e \(\mu\) tali che:
\[
    \vec{c} = \lambda \vec{a} + \mu \vec{b}
\]

\begin{tikzpicture}
    \draw[thick, ->] (0,0) -- (2,1) node[above] {$\vec{a}$};
    \draw[thick, ->] (0,0) -- (4,2) node[right] {$\vec{b} = 2\vec{a}$};
    \draw[thick, ->] (0,0) -- (1,2) node[right] {$\vec{c}$};
    \draw[thick, ->] (0,0) -- (3,6) node[above] {$\vec{d} = 3\vec{c}$};
\end{tikzpicture}

\newpage
\subsection{Il prodotto scalare, definizione, calcolo e applicazioni}

\subsubsection{Definizione di prodotto scalare}
Il prodotto scalare di due vettori \(\vec{a} = (a_1, a_2, a_3)\) e
\(\vec{b} = (b_1, b_2, b_3)\) è definito come:
\[
    \vec{a} \cdot \vec{b} = a_1 b_1 + a_2 b_2 + a_3 b_3
\]
A parole, il prodotto scalare è una misura di quanto due vettori puntano nella stessa
direzione.\\
È calcolato moltiplicando le componenti corrispondenti dei vettori e sommando i risultati.


\subsubsection{Calcolo del prodotto scalare}
Per i vettori \(\vec{a} = (1, 2, 3)\) e \(\vec{b} = (4, -5, 6)\), il prodotto scalare è:
\[
    \vec{a} \cdot \vec{b} = 1 \cdot 4 + 2 \cdot (-5) + 3 \cdot 6 = 4 - 10 + 18 = 12
\]

\subsubsection{Applicazioni del prodotto scalare}
Il prodotto scalare è utile per determinare l'angolo \(\theta\) tra due vettori:
\[
    \vec{a} \cdot \vec{b} = \lvert\vec{a}\rvert \cdot \lvert\vec{b}\rvert \cos \theta
\]
Se \(\vec{a} \cdot \vec{b} = 0\), i vettori sono ortogonali.

\begin{tikzpicture}
    \draw[thick, ->] (0,0) -- (3,0) node[right] {$\vec{a}$};
    \draw[thick, ->] (0,0) -- (1,2) node[above] {$\vec{b}$};
    \draw (1,0) arc (0:63:1) node[midway, right] {$\theta$};
\end{tikzpicture}

\newpage
\subsection{Il prodotto vettoriale, definizione, calcolo e applicazioni}

\subsubsection{Definizione di prodotto vettoriale}
Il prodotto vettoriale di due vettori \(\vec{a}\) e \(\vec{b}\) è un vettore \(\vec{c}\)
tale che:
\[
    \vec{c} = \vec{a} \times \vec{b}
\]
e ha una direzione perpendicolare sia a \(\vec{a}\) che a \(\vec{b}\) secondo la regola
della mano destra.

A parole, il prodotto vettoriale è un vettore che è perpendicolare a entrambi i vettori
originali e il cui modulo rappresenta l'area del parallelogramma formato dai due vettori.

\subsubsection{Calcolo del prodotto vettoriale}
Per i vettori \(\vec{a} = (a_1, a_2, a_3)\) e \(\vec{b} = (b_1, b_2, b_3)\),
il prodotto vettoriale è dato da:
\[
    \vec{a} \times \vec{b} = (a_2 b_3 - a_3 b_2, a_3 b_1 - a_1 b_3, a_1 b_2 - a_2 b_1)
\]
Ad esempio, per \(\vec{a} = (1, 2, 3)\) e \(\vec{b} = (4, 5, 6)\):
\[
    \vec{a} \times \vec{b} = (2 \cdot 6 - 3 \cdot 5, 3 \cdot 4 - 1 \cdot 6, 1 \cdot 5 - 2 \cdot 4) = (12 - 15, 12 - 6, 5 - 8) = (-3, 6, -3)
\]

\subsubsection{Applicazioni del prodotto vettoriale}
Il prodotto vettoriale è utile per determinare il vettore normale a un piano definito da due
vettori.

\begin{tikzpicture}
    \draw[thick, ->] (0,0,0) -- (2,0,0) node[right] {$\vec{a}$};
    \draw[thick, ->] (0,0,0) -- (0,2,0) node[above] {$\vec{b}$};
    \draw[thick, ->] (0,0,0) -- (0,0,2) node[below] {$\vec{a} \times \vec{b}$};
\end{tikzpicture}

\newpage
\subsection{Equazione parametrica della retta nello spazio, posizione reciproca tra rette}

\subsubsection{Equazione parametrica della retta}
L'equazione parametrica di una retta nello spazio è espressa in funzione di un parametro
\( t \). Se un punto \( \vec{r}_0 = (x_0, y_0, z_0) \) appartiene alla retta e
\( \vec{d} = (d_x, d_y, d_z) \) è il vettore direzionale della retta,
l'equazione parametrica è data da:
\[
    \vec{r}(t) = \vec{r}_0 + t \vec{d}
\]
In forma coordinata, le equazioni parametriche si scrivono come:
\[
    \begin{cases}
        x = x_0 + t d_x \\
        y = y_0 + t d_y \\
        z = z_0 + t d_z
    \end{cases}
\]
dove \( t \) è un parametro reale.

\subsubsection{Esempio di equazione parametrica}
Supponiamo che la retta passi per il punto \( (1, 2, 3) \) e abbia un vettore direzionale \( (4, -2, 1) \). L'equazione parametrica della retta è:
\[
    \begin{cases}
        x = 1 + 4t \\
        y = 2 - 2t \\
        z = 3 + t
    \end{cases}
\]

\subsubsection{Posizione reciproca tra rette}
La posizione reciproca tra due rette nello spazio può essere classificata come segue:
\vspace*{0.2cm}
\begin{center}
    \begin{tikzpicture}[
        level 1/.style = {sibling distance = 4cm},
        level 2/.style = {sibling distance = 3cm},
        level 3/.style = {sibling distance = 3cm},
        level 4/.style = {sibling distance = 3cm}
    ]
    \node {\Ovalbox{Rette nello spazio}}
        child {
            node {\ovalbox{
                    \begin{minipage}{.08\textwidth}
                        \centering Sghembe
                    \end{minipage}
                }
            }
        }
        child {
            node {\ovalbox{
            \begin{minipage}{.11\textwidth}
                \centering Complanari
            \end{minipage}
                }
            }
            child {
                node {\ovalbox{
                        \begin{minipage}{.078\textwidth}
                            \centering Incidenti
                        \end{minipage}
                    }
                }
            }
            child {
                node {\ovalbox{
                        \begin{minipage}{.078\textwidth}
                            \centering Parallele
                        \end{minipage}
                    }
                }
                child {
                    node {\ovalbox{
                        \begin{minipage}{.07\textwidth}
                            \centering Distinte
                        \end{minipage}
                        }
                    }
                }
                child {
                    node {\ovalbox{
                        \begin{minipage}{.105\textwidth}
                            \centering Coincidenti
                        \end{minipage}
                        }
                    }
                }
            }
        };
    \end{tikzpicture}
\end{center}

\paragraph{Rette incidenti} \phantom{}\\
Due rette sono incidenti se hanno un punto in comune. Per verificare se due rette sono
incidenti, si risolvono i loro sistemi di equazioni parametriche per trovare un punto di
intersezione comune.

\paragraph{Rette parallele} \phantom{}\\
Due rette sono parallele se i loro vettori direzionali sono proporzionali.
Se \( \vec{d}_1 \) e \( \vec{d}_2 \) sono i vettori direzionali delle due rette,
allora le rette sono parallele se esiste un numero reale \(\lambda\) tale che:
\[
    \vec{d}_1 = \lambda \vec{d}_2
\]

\paragraph{Rette sghembe} \phantom{}\\
Due rette sono sghembe se non sono né incidenti né parallele. Questo significa che non hanno
punti in comune e non giacciono sullo stesso piano.

\newpage
\subsubsection{Esempi di posizione reciproca tra rette}
Consideriamo le seguenti rette:
\[
    r \begin{cases}
        x = t \\
        y = 1 - t \\
        z = t
    \end{cases}
\quad
    s \begin{cases}
        x = 4 + 2s \\
        y = -1 - s \\
        z = 2 + s
    \end{cases}
\]

\paragraph{Verifica di parallelismo} \phantom{}\\
I vettori direzionali sono \( \vec{d}_1 = (1, -1, 1) \) e \( \vec{d}_2 = (2, -1, 1) \).
Poiché \( \vec{d}_1 \neq \lambda \vec{d}_2 \), le rette non sono parallele.

\paragraph{Verifica di intersezione} \phantom{}\\
Per verificare se le rette sono incidenti, eguagliamo le equazioni parametriche:
\[
    \begin{cases}
        t = 4 + 2s \\
        1 - t = -1 - s \\
        t = 2 + s
    \end{cases}
\]
Risolvendo il sistema si trova \(t = 0\) e \(s=1\).
Usando la seconda equazione per verificare che le due rette siano incidenti, risulta che
\(1=1\), dunque le due rette \underline{sono incidenti}.
\vspace*{1cm}
\begin{center}
    \begin{tikzpicture}
        \begin{axis}[
            view={50}{40},
            axis lines=center,
            xlabel={$x$},
            ylabel={$y$},
            zlabel={$z$},
            grid=major,
            xmin=-2, xmax=6,
            ymin=-3, ymax=2,
            zmin=-1, zmax=3,
            width=12cm,
            height=12cm,
            ticklabel style={font=\small},
            label style={font=\small},
            xtick=\empty,
            ytick=\empty,
            ztick=\empty,
        ]
        % Retta r
        \addplot3[domain=-5:5, samples=50, thick, blue] ({x}, {1-x}, {x});
        \node[above right, blue] at (axis cs: 2,-1,2) {$r$};
        
        % Retta s
        \addplot3[domain=20:-20, samples=50, thick, red] ({4+2*x}, {-1-x}, {2+x});
        \node[below left, red] at (axis cs: 6,-2,3) {$s$};
        
        % Punto di intersezione
        \addplot3[only marks, mark=*] coordinates {(0,1,0)};
        \node at (1.15,1.15,0.3) {(0,1,0)};
        \end{axis}
    \end{tikzpicture}
\end{center}

\newpage
\subsection{Equazione cartesiana del piano nello spazio}

\subsubsection{Definizione dell'equazione cartesiana}
L'equazione cartesiana di un piano nello spazio è un'equazione lineare nelle variabili
\(x\), \(y\) e \(z\). È data da:
\[
    ax + by + cz + d = 0
\]
dove \(a\), \(b\), \(c\) e \(d\) sono costanti. I coefficienti \(a\), \(b\) e \(c\)
rappresentano le componenti del vettore normale al piano, \(\vec{n} = (a, b, c)\).

A parole, l'equazione cartesiana di un piano nello spazio descrive tutti i punti
\((x, y, z)\) che giacciono su un piano specificato da un punto noto e un vettore normale.
Il vettore normale è perpendicolare alla superficie del piano e definisce l'orientamento del
piano stesso.

\subsubsection{Determinazione dell'equazione di un piano}
Per determinare l'equazione cartesiana di un piano, è necessario conoscere un punto
\((x_0, y_0, z_0)\) sul piano e il vettore normale \(\vec{n} = (a, b, c)\).
L'equazione del piano può essere trovata sostituendo le coordinate del punto e del vettore
normale nell'equazione:
\[
    a(x - x_0) + b(y - y_0) + c(z - z_0) = 0
\]
Espandendo e semplificando si ottiene:
\[
    ax + by + cz + d = 0
\]
dove \(d = -(ax_0 + by_0 + cz_0)\).

\subsubsection{Esempio di determinazione dell'equazione di un piano}
Supponiamo di avere un punto \((2, -1, 3)\) e un vettore normale \(\vec{n} = (1, 2, -1)\). L'equazione del piano è:
\[
    1(x - 2) + 2(y + 1) - 1(z - 3) = 0
\]
Espandendo e semplificando:
\[
    x + 2y - z + 3 = 0
\]

\subsubsection{Visualizzazione di un piano}
Il grafico seguente mostra il piano \(x + 2y - z + 3 = 0\) nello spazio tridimensionale, con il vettore normale \(\vec{n} = (1, 2, -1)\) e il punto \((2, -1, 3)\) sul piano.
\vspace*{0.5cm}
\begin{center}
    \begin{tikzpicture}
        \begin{axis}[
            view={60}{30},
            axis lines=center,
            xlabel={$x$},
            ylabel={$y$},
            zlabel={$z$},
            grid=major,
            xmin=-3, xmax=3,
            ymin=-3, ymax=3,
            zmin=-3, zmax=3,
            width=10cm,
            height=10cm,
            ticklabel style={font=\small},
            label style={font=\small},
            xtick=\empty,
            ytick=\empty,
            ztick=\empty,
        ]
        % Draw the plane
        \addplot3[surf, opacity=0.7, domain=-3:3, y domain=-3:3] {x + 2*y + 3};
        % Draw the normal vector
        \addplot3[->, thick, blue] coordinates {(2,-1,3) (3,-1,2)};
        \node[below, blue] at (axis cs: 3,-1,2) {$\vec{n}$};
        % Draw the point
        \addplot3[only marks, mark=*] coordinates {(2,-1,3)} node[right] {$(2, -1, 3)$};
        % Origin
        \node[below left] at (axis cs: 0,0,0) {$(0,0,0)$};
        \end{axis}
    \end{tikzpicture}
\end{center}

\newpage
\subsection{Semplici problemi di intersezione, di distanza e di angoli tra punti, rette o piani}

\subsubsection{Problemi di intersezione}
Per trovare il punto di intersezione tra una retta e un piano, si sostituisce l'equazione
parametrica della retta nell'equazione cartesiana del piano e si risolve per il parametro.

A parole, l'intersezione tra due entità geometriche nello spazio (come rette o piani) è il
punto o la linea comune a entrambe. Determinare l'intersezione significa trovare questi punti
o linee comuni.

\subsubsection{Esempio di intersezione tra una retta e un piano}
Supponiamo di avere una retta data da:
\[
    \vec{r}(t) = \begin{pmatrix} 1 \\ 2 \\ 3 \end{pmatrix}
    + t \begin{pmatrix} 1 \\ -1 \\ 2 \end{pmatrix}
\]
e un piano dato da:
\[
    x + 2y - z - 1 = 0
\]
Per trovare il punto di intersezione, sostituiamo le coordinate parametriche della retta
nell'equazione del piano:
\[
    (1 + t) + 2(2 - t) - (3 + 2t) - 1 = 0
\]
Risolviamo per \(t\):
\[
    1 + t + 4 - 2t - 3 - 2t - 1 = 0 \implies -2t + 1 = 0 \implies t = \frac{1}{2}
\]
Il punto di intersezione è:
\[
    \vec{r}\left(\frac{1}{2}\right) = \begin{pmatrix} 1 \\ 2 \\ 3 \end{pmatrix}
    + \frac{1}{2} \begin{pmatrix} 1 \\ -1 \\ 2 \end{pmatrix} =
    \begin{pmatrix} 1.5 \\ 1.5 \\ 4 \end{pmatrix}
\]

\subsubsection{Problemi di distanza}
La distanza tra un punto e un piano può essere trovata usando la formula:
\[
    d = \frac{|ax_1 + by_1 + cz_1 + d|}{\sqrt{a^2 + b^2 + c^2}}
\]

A parole, la distanza tra un punto e un'entità geometrica (come una retta o un piano) è la
lunghezza del segmento di linea perpendicolare che li unisce. È la misura più breve tra il
punto e l'entità.

\subsubsection{Esempio di distanza tra un punto e un piano}
Supponiamo di avere un punto \(P(1, 2, 3)\) e un piano dato da \(x + 2y - z - 1 = 0\).
La distanza è:
\[
    d = \frac{|1 \cdot 1 + 2 \cdot 2 - 1 \cdot 3 - 1|}{\sqrt{1^2 + 2^2 + (-1)^2}} =
    \frac{|1 + 4 - 3 - 1|}{\sqrt{1 + 4 + 1}} = \frac{1}{\sqrt{6}}
\]

\subsubsection{Problemi di angoli}
L'angolo tra due rette può essere trovato usando il prodotto scalare dei loro vettori
direzionali, mentre l'angolo tra due piani può essere trovato usando il prodotto scalare dei
loro vettori normali.

A parole, l'angolo tra due entità geometriche (come rette o piani) è la misura
dell'inclinazione tra di esse. Per le rette, è l'angolo tra i loro vettori direzionali.
Per i piani, è l'angolo tra i loro vettori normali.

\subsubsection{Esempio di angolo tra due rette}
Supponiamo di avere due rette con vettori direzionali \(\vec{d}_1 = (1, 2, 3)\) e \(\vec{d}_2 = (4, -5, 6)\). L'angolo \(\theta\) tra le rette è dato da:
\[
    \cos \theta = \frac{\vec{d}_1 \cdot \vec{d}_2}{|\vec{d}_1||\vec{d}_2|} =
    \frac{1 \cdot 4 + 2 \cdot (-5) + 3 \cdot 6}{\sqrt{1^2 + 2^2 + 3^2}
    \sqrt{4^2 + (-5)^2 + 6^2}} = \frac{12}{\sqrt{1078}}
\]
\[
    \theta = \cos^{-1} \left( \frac{12}{\sqrt{1078}} \right)
\]

\subsubsection{Visualizzazione di problemi geometrici}
Il grafico seguente mostra una retta e un piano, illustrando l'intersezione e la distanza tra
un punto e il piano.

\begin{tikzpicture}
    \begin{axis}[
        view={60}{30},
        axis lines=center,
        xlabel={$x$},
        ylabel={$y$},
        zlabel={$z$},
        grid=major,
        xmin=-3, xmax=5,
        ymin=-3, ymax=5,
        zmin=-1, zmax=5,
        width=10cm,
        height=10cm,
        ticklabel style={font=\small},
        label style={font=\small},
        xtick=\empty,
        ytick=\empty,
        ztick=\empty,
    ]
    % Draw the plane
    \addplot3[surf, opacity=0.7, domain=-3:3, y domain=-3:3] {x + 2*y - 1};
    % Draw the line
    \addplot3[domain=-1:3, samples=50, thick, blue] ({1 + x}, {2 - x}, {3 + 2*x});
    \node[below left, blue] at (axis cs: 2.5,0.5,7) {Retta $r$};
    % Intersection point
    \addplot3[only marks, mark=*] coordinates {(1.5,1.5,4)} node[below right] {$(1.5, 1.5, 4)$};
    % Point
    \addplot3[only marks, mark=*] coordinates {(1,2,3)} node[below right] {$(1, 2, 3)$};
    % Origin
    \node[left] at (axis cs: 0,0,0) {$(0,0,0)$};
    \end{axis}
\end{tikzpicture}

\newpage
\part{Trigonometria}

\section{Funzioni trigonometriche}

\subsection{Definizione delle funzioni trigonometriche}

\subsubsection{Funzioni seno e coseno}
Le funzioni seno e coseno sono definite come rapporti di lunghezze in un triangolo rettangolo.
Per un angolo \(\theta\), il seno è il rapporto tra il cateto opposto e l'ipotenusa, mentre il
coseno è il rapporto tra il cateto adiacente e l'ipotenusa:
\[
    \sin \theta = \frac{\text{cateto opposto}}{\text{ipotenusa}},
    \quad \cos \theta = \frac{\text{cateto adiacente}}{\text{ipotenusa}}
\]

\subsubsection{Funzione tangente}
La funzione tangente è definita come il rapporto tra il seno e il coseno dello stesso angolo:
\[
    \tan \theta = \frac{\sin \theta}{\cos \theta}
\]

\subsubsection{Funzioni trigonometriche nel cerchio unitario}
Le funzioni trigonometriche possono essere estese agli angoli oltre \(0\) e \(90^\circ\)
utilizzando il cerchio unitario. Per un angolo \(\theta\) nel cerchio unitario:
\[
    \sin \theta = y, \quad \cos \theta = x, \quad \tan \theta = \frac{y}{x}
\]
\begin{center}
    \begin{tikzpicture}[scale=3.75]
        \definecolor{darkgreen}{rgb}{0.0, 0.7, 0.0}
        % Draw the x and y axes
        \draw[thick, ->] (-1.5,0) -- (1.5,0) node[right] {\(x\)};
        \draw[thick, ->] (0,-1.5) -- (0,1.5) node[above] {\(y\)};
    
        % Draw the unit circle
        \draw (0,0) circle (1);
    
        % Draw the angle theta
        \draw[thick, ->, blue] (0,0) -- (0.866,0.5) node[anchor=south west] {\(P(x,y)\)};
    
        % Draw dashed lines to indicate the projections on the axes
        \draw[dashed] (0.866,0) -- (0.866,0.5) -- (0,0.5);
    
        % Label the angle theta
        \draw (0.3,0) arc (0:30:0.3);
        \node at (0.35,0.10) {\(\theta\)};
    
        % Label the origin
        \node at (-0.05, -0.05) {O};
    
        % Draw the projections on the axes
        \draw[thick, red] (0,0) -- (0.866, 0) node[midway, below] {\(\cos \theta\)};
        \draw[thick, darkgreen] (0,0) -- (0, 0.5) node[midway, left] {\(\sin \theta\)};
    
        % Additional labels and lines
        \node at (1.075, 0.05) {1};
        \node at (-0.05, 1.1) {1};
        \node at (-1.075, 0.05) {-1};
        \node at (-0.05, -1.1) {-1};
    
        % Draw dashed unit distances
        \draw[dashed] (0,0) -- (-1,0);
        \draw[dashed] (0,0) -- (0,-1);
    
        % Labels for each quadrant
        \node at (1, 1) {I};
        \node at (-1, 1) {II};
        \node at (-1, -1) {III};
        \node at (1, -1) {IV};
    \end{tikzpicture}
\end{center}

\newpage
\subsection{Grafici e proprietà, funzioni trigonometriche inverse}

\subsubsection{Grafico della funzione seno}
Il grafico della funzione seno è una curva periodica che oscilla tra -1 e 1, con un periodo
di \(2\pi\). Il seno di 0 è 0, e il seno di \(\pi/2\) è 1.

\begin{tikzpicture}
    \begin{axis}[
        axis lines = middle,
        xlabel = \(x\),
        ylabel = {\(\sin x\)},
        ymin=-1.5, ymax=1.5,
        xtick={-3.14, -1.57, 0, 1.57, 3.14},
        xticklabels={$-\pi$, $-\frac{\pi}{2}$, $0$, $\frac{\pi}{2}$, $\pi$},
        ytick={-1, 0, 1},
        grid=major,
        width=12cm,
        height=6cm
    ]
    \addplot[domain=-3.14:3.14, samples=100, blue, thick] {sin(deg(x))};
    \end{axis}
\end{tikzpicture}

\subsubsection{Grafico della funzione coseno}
Il grafico della funzione coseno è simile a quello del seno, ma è traslato di \(\pi/2\) a
sinistra. Oscilla tra -1 e 1, con un periodo di \(2\pi\).

\begin{tikzpicture}
    \begin{axis}[
        axis lines = middle,
        xlabel = \(x\),
        ylabel = {\(\cos x\)},
        ymin=-1.5, ymax=1.5,
        xtick={-3.14, -1.57, 0, 1.57, 3.14},
        xticklabels={$-\pi$, $-\frac{\pi}{2}$, $0$, $\frac{\pi}{2}$, $\pi$},
        ytick={-1, 0, 1},
        grid=major,
        width=12cm,
        height=6cm
    ]
    \addplot[domain=-3.14:3.14, samples=100, red, thick] {cos(deg(x))};
    \end{axis}
\end{tikzpicture}

\subsubsection{Funzioni trigonometriche inverse}
Le funzioni trigonometriche inverse sono utilizzate per trovare l'angolo dato il valore di
una funzione trigonometrica. Le funzioni inverse principali sono l'arcoseno, l'arcocoseno e
l'arcotangente:
\[
    y = \sin^{-1} x, \quad y = \cos^{-1} x, \quad y = \tan^{-1} x
\]

\begin{tikzpicture}
    \begin{axis}[
        axis lines = middle,
        xlabel = \(x\),
        ylabel = {\(\sin^{-1} x\)},
        ymin=-1.7, ymax=1.7,
        xtick={-1, 0, 1},
        ytick={-1.57, 0, 1.57},
        yticklabels={$-\frac{\pi}{2}$, $0$, $\frac{\pi}{2}$},
        grid=major,
        width=12cm,
        height=6cm,
        domain=-1:1
    ]
    \addplot[domain=-1:1, samples=100, green, thick] {rad(asin(x))};
    \end{axis}
\end{tikzpicture}

\newpage
\subsection{Applicazione al triangolo rettangolo}

\subsubsection{Teorema di Pitagora}
Il teorema di Pitagora stabilisce che in un triangolo rettangolo, il quadrato dell'ipotenusa
è uguale alla somma dei quadrati dei cateti:
\[
    c^2 = a^2 + b^2
\]

\begin{tikzpicture}[scale=0.8]
    % Define coordinates
    \coordinate (A) at (0, 0);
    \coordinate (B) at (3, 0);
    \coordinate (C) at (0, 3);

    % Draw the triangle
    \draw[thick] (A) -- (B) -- (C) -- cycle;

    % Draw squares on the legs
    \draw[fill=blue!20] (A) -- ++(-3,0) -- ++(0,3) -- ++(3,0) -- cycle;
    \draw[fill=red!20] (B) -- ++(-3,0) -- ++(0,-3) -- ++(3,0) -- cycle;
    \draw[fill=purple!20] (C) -- ++(3,-3) -- ++(3,3) -- ++(-3,3) -- cycle;

    % Draw the triangle again to make sure it is on top
    \draw[thick] (A) -- (B) -- (C) -- cycle;

    % Label the sides
    \node at (1.5, -0.3) {$a$};
    \node at (-0.3, 1.5) {$b$};
    \node at (1.75, 1.75) {$c$};

    % Label the squares
    \node at (1.5, -1.5) {$a^2$};
    \node at (-1.5, 1.5) {$b^2$};
    \node at (3, 3) {$c^2$};

    % Right angle symbol
    \draw (A) rectangle +(0.3,0.3);
\end{tikzpicture}

\subsubsection{Rapporti trigonometrici nel triangolo rettangolo}
In un triangolo rettangolo, i rapporti trigonometrici delle funzioni seno, coseno e tangente
si applicano come segue:
\[
    \sin \theta = \frac{\text{cateto opposto}}{\text{ipotenusa}},
    \quad \cos \theta = \frac{\text{cateto adiacente}}{\text{ipotenusa}},
    \quad \tan \theta = \frac{\text{cateto opposto}}{\text{cateto adiacente}}
\]

\subsubsection{Esempi di applicazione}
Calcolare i lati di un triangolo rettangolo dati uno degli angoli acuti e un lato.
Ad esempio, se \(\theta = 30^\circ\) e l'ipotenusa è 10, allora:
\[
    \sin 30^\circ = \frac{1}{2} = \frac{\text{cateto opposto}}{10} \implies
    \text{cateto opposto} = 5
\]
\[
    \cos 30^\circ = \frac{\sqrt{3}}{2} = \frac{\text{cateto adiacente}}{10} \implies
    \text{cateto adiacente} = 5\sqrt{3}
\]

\newpage
\subsection{Risoluzione di equazioni del tipo f(x) = k}
\paragraph*{(dove \(f\) è una funzione trigonometrica e \(k \in \mathbb{R}\))}

\subsubsection{Equazioni con la funzione seno}
Per risolvere equazioni del tipo \(\sin x = k\), si considerano i valori di \(k\) che sono
compresi tra -1 e 1. L'equazione ha soluzioni se \(-1 \leq k \leq 1\):
\[
    x = \sin^{-1} k + 2n\pi, \quad x = \pi - \sin^{-1} k + 2n\pi, \quad n \in \mathbb{Z}
\]

\subsubsection{Equazioni con la funzione coseno}
Per risolvere equazioni del tipo \(\cos x = k\), si considerano i valori di \(k\) compresi
tra -1 e 1. L'equazione ha soluzioni se \(-1 \leq k \leq 1\):
\[
    x = \cos^{-1} k + 2n\pi, \quad x = -\cos^{-1} k + 2n\pi, \quad n \in \mathbb{Z}
\]

\subsubsection{Equazioni con la funzione tangente}
Per risolvere equazioni del tipo \(\tan x = k\), si considerano tutti i valori reali
di \(k\). L'equazione ha soluzioni per tutti i valori di \(k\):
\[
    x = \tan^{-1} k + n\pi, \quad n \in \mathbb{Z}
\]

\subsubsection{Esempi di risoluzione}
Risolviamo l'equazione \(\sin x = \frac{1}{2}\):
\[
    x = \sin^{-1} \frac{1}{2} + 2n\pi = \frac{\pi}{6} + 2n\pi,
    \quad x = \pi - \sin^{-1} \frac{1}{2} + 2n\pi = \frac{5\pi}{6} + 2n\pi,
    \quad n \in \mathbb{Z}
\]











\end{document}
