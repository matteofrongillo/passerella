\documentclass{article}
\usepackage{frongillo}
\pgfplotsset{compat=1.18}

\title{English - Mock exam\\Passerella 2023-24}
\author{Matteo Frongillo}

\begin{document}

\maketitle
\tableofcontents
\pagebreak

\section{A View from the Bridge}

\subsection{Trama del libro}

\subsubsection{Atto primo}
L'azione si apre con una scena della strada e la facciata dell'appartamento di
Eddie Carbone, scaricatore portuale di circa quarant'anni, duro e protettivo.
L'avvocato Alfieri, che funge da narratore, introduce la storia e i temi di
giustizia e legge nel contesto della comunità italo-americana.

Eddie è profondamente affezionato alla nipote diciassettenne Catherine e mostra
gelosia quando lei annuncia di aver ottenuto un lavoro come stenografa.
Nonostante le obiezioni di Eddie, Beatrice e Catherine insistono sul fatto che
il lavoro è una buona opportunità per Catherine.

L'atmosfera si complica ulteriormente quando arrivano i cugini di Beatrice,
Marco e Rodolfo, immigrati clandestinamente dall'Italia. Marco è un uomo robusto
e serio, venuto in America per lavorare e sostenere la sua famiglia in Italia.
Rodolfo è giovane, biondo e spensierato, con ambizioni di rimanere in America e
fare fortuna.

Eddie accoglie i cugini con una calorosa ospitalità, ma presto emerge la
tensione quando Catherine e Rodolfo cominciano a sviluppare un interesse
romantico reciproco. Eddie diventa sempre più infastidito e sospettoso,
convinto che Rodolfo stia usando Catherine per ottenere la cittadinanza
americana. La gelosia e la protezione ossessiva di Eddie nei confronti di
Catherine diventano evidenti e disturbano la pace familiare.

\subsubsection{Atto Secondo}
L'atto secondo si concentra sulla crescente tensione tra Eddie e Rodolfo. Eddie
cerca in tutti i modi di screditare Rodolfo, facendo insinuazioni sulla sua 
sessualità e sui suoi veri motivi. La situazione raggiunge un punto critico
quando Eddie bacia Catherine in un momento di confusione e frustrazione,
rivelando in maniera ancora più chiara i suoi sentimenti inappropriati.

Beatrice, che è stata paziente ma preoccupata, affronta Eddie sulla sua
ossessione per Catherine, esortandolo a lasciare che la ragazza viva la sua
vita. Eddie però si rifiuta di accettare la relazione tra Catherine e Rodolfo
e decide di denunciarli all'ufficio immigrazione, sperando che questo costringa
Rodolfo a lasciare Catherine.

La denuncia di Eddie provoca un'inevitabile tragedia. Marco, sentendosi
tradito, accusa Eddie di aver violato il codice d'onore della loro comunità.
La tensione culmina in una lotta tra Marco ed Eddie, durante la quale Eddie
viene ucciso con il proprio coltello.

Alfieri conclude la storia riflettendo sulla natura tragica degli eventi e
sull'ineluttabilità del destino di Eddie, sottolineando la complessità della
giustizia e del sacrificio in una comunità strettamente legata alle proprie
tradizioni.

\subsection{Temi Principali}

\subsubsection{Giustizia e Legge}
Alfieri rappresenta la tensione tra la legge ufficiale e il codice d'onore della
comunità.
\subsubsection{Onore e Tradizione}
Il conflitto tra le nuove generazioni e le vecchie tradizioni italo-americane.
\subsubsection{Desiderio e Gelosia}
La relazione complessa e disturbante di Eddie verso Catherine.
\subsubsection{Immigrazione e Sogno Americano}
Le speranze e le difficoltà degli immigrati in cerca di una vita migliore.

La trama, con la sua intensa drammaticità e i suoi temi universali,
fa di A View from the Bridge un'opera potente che esplora le profondità
della natura umana e delle dinamiche familiari.

\subsection{Personaggi principali}

\subsubsection{Eddie Carbone}
Eddie Carbone è il protagonista della storia, un uomo sulla quarantina che lavora come
scaricatore di porto a Brooklyn. È descritto come duro e appesantito, ma profondamente
legato alla sua famiglia. Eddie è un uomo onesto, ma è anche ossessionato dalla protezione
della sua nipote adottiva Catherine. La sua gelosia e i sentimenti inappropriati nei confronti
di Catherine lo portano a compiere azioni distruttive che culminano nella sua tragica fine.
La sua incapacità di riconoscere e controllare i suoi desideri interiori lo rende un
personaggio complesso e tragico.

\subsubsection{Beatrice Carbone}
Beatrice è la moglie di Eddie e la cugina dei due immigrati, Marco e Rodolfo. È una donna
premurosa e comprensiva che cerca di mantenere l'armonia nella famiglia. Beatrice è
consapevole dei sentimenti problematici di Eddie verso Catherine e tenta di farglieli
riconoscere, ma spesso senza successo. È leale e amorevole, ma anche preoccupata per
l'integrità della sua famiglia.

\subsubsection{Catherine}
Catherine è la nipote diciassettenne di Eddie e Beatrice. È giovane, attraente e desiderosa
di indipendenza. Catherine è entusiasta della sua nuova opportunità di lavoro come stenografa,
ma è anche molto legata a Eddie, vedendolo come una figura paterna. La sua relazione con
Rodolfo e la reazione di Eddie a questa relazione sono al centro del conflitto della storia.
Catherine è innocente e ingenua, ma dimostra anche un desiderio di crescere e prendere il
controllo della sua vita.

\subsubsection{Marco}
Marco è uno dei cugini di Beatrice, un uomo di 32 anni robusto e serio che viene dall'Italia
per lavorare e sostenere la sua famiglia. Marco è rispettoso e riconoscente verso Eddie per
averlo accolto. È un lavoratore instancabile, motivato dal desiderio di aiutare i suoi cari.
La sua lealtà alla famiglia e il suo codice d'onore lo portano a confrontarsi con Eddie quand
 si sente tradito, culminando in un violento scontro.

\subsubsection{Rodolfo}
Rodolfo è il fratello più giovane di Marco, un giovane biondo e vivace con aspirazioni
artistiche. È affascinante e spensierato, con un talento per il canto e un sogno di rimanere
in America per fare fortuna. Rodolfo sviluppa una relazione romantica con Catherine, suscitando
la gelosia di Eddie. Sebbene Rodolfo sia visto da Eddie come opportunista, è genuinamente
affezionato a Catherine e desidera costruirsi una vita migliore.

\subsubsection{Alfieri}
Alfieri è un avvocato di mezza età che funge da narratore e coro greco della storia. È un uomo
riflessivo e dignitoso che rappresenta la legge e l'ordine. Alfieri fornisce commenti sullo
sviluppo della trama e sui dilemmi morali dei personaggi. Egli comprende le tensioni tra la
legge ufficiale e il codice d'onore della comunità, e cerca di consigliare Eddie senza successo.

\subsubsection{Personaggi Minori}
Louis e Mike: Scaricatori del porto e amici di Eddie. Rappresentano il contesto lavorativo e
sociale di Eddie. Le loro conversazioni e interazioni con Eddie contribuiscono a dipingere il
quadro della vita di Red Hook.
\\
Tony Berelli: Un personaggio che appare brevemente per informare Eddie dell'arrivo dei cugini
Marco e Rodolfo.

\subsection{Analisi del libro}

\subsubsection{Tipo di narratore}
Il narratore è un personaggio interno alla storia che racconta gli eventi in prima persona,
parlando direttamente al pubblico e offrendo il suo punto di vista. Utilizza una combinazione
di focalizzazione esterna e interna: descrive eventi e comportamenti in modo obiettivo e
riflette sui pensieri e sentimenti dei personaggi, permettendo al pubblico di comprendere le
loro motivazioni.

Questo narratore è onnisciente, conoscendo tutto ciò che accade e le dinamiche tra i personaggi.
Può anticipare sviluppi futuri e offrire una visione completa della storia. È considerato
affidabile, e le sue osservazioni e giudizi sono percepiti come credibili e autorevoli.

Svolge anche una funzione simile a quella del coro nelle tragedie greche, offrendo commenti
morali e filosofici sugli eventi. Questo aiuta il pubblico a contestualizzare gli eventi nel
quadro più ampio della condizione umana e delle leggi sociali.

\subsubsection{Ambientazione}
L'ambientazione principale è Red Hook, un quartiere portuale di Brooklyn, New York,
durante gli anni '50. La casa di Eddie Carbone è il fulcro delle interazioni familiari e
sociali, rappresentando sia un rifugio che un campo di battaglia per i conflitti personali.
Red Hook, con i suoi carichi di navi e le sue strade trafficate, riflette il duro lavoro e
la lotta quotidiana degli immigrati italiani che vi risiedono.

\subsubsection{Contestualizzazione storica}
È ambientato negli anni '50, un periodo di grande immigrazione italiana negli Stati Uniti.
Questa era fu caratterizzata da significative tensioni sociali e culturali, poiché molti
immigrati cercavano di mantenere le loro tradizioni e valori mentre si adattavano alla vita
americana. Red Hook, un quartiere prevalentemente operaio, rappresenta una microcosmo di queste
dinamiche.

Gli immigrati italiani, come i personaggi di Marco e Rodolfo, arrivavano negli Stati Uniti in
cerca di opportunità economiche e di un futuro migliore per le loro famiglie. Tuttavia,
affrontavano anche discriminazione, sfruttamento e dure condizioni di lavoro. La legge
sull'immigrazione e le preoccupazioni per la deportazione erano costanti minacce per molti
di loro, come illustrato nel dramma con la denuncia di Eddie all'ufficio immigrazione.

La figura di Alfieri riflette la complessità del sistema legale americano e le sfide che gli
immigrati dovevano affrontare nell'integrare le proprie norme culturali con quelle della loro
nuova patria. La sua funzione di narratore e mediatore legale sottolinea le tensioni tra
giustizia personale e giustizia istituzionale, un tema centrale nel dramma.












\end{document}