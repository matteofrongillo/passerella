\documentclass{article}

\usepackage[fleqn]{amsmath}
\usepackage{amssymb}
\usepackage{hyperref}
\usepackage{url}
\usepackage{graphicx}
\usepackage{geometry}
\usepackage[italian]{babel}
\usepackage{enumitem}
\usepackage{parskip}
\usepackage{chemfig}
\usepackage{pdfpages}
\usepackage{pgfplots}
\pgfplotsset{compat=1.17}
\usepackage{xcolor}
\usepackage{tikz}
\usepackage{fancybox}
\usepackage{makecell}
\usepackage{soul}
\usepackage{ulem}
\usepackage{wrapfig}
\usepackage{subcaption}
\usepackage{svg}

\usetikzlibrary{shapes.geometric, arrows}
\usetikzlibrary{decorations.pathreplacing}
\usetikzlibrary{arrows.meta}
\usetikzlibrary{positioning}

\geometry{    
    a4paper,    
    total={170mm, 257mm},    
    left=20mm,    
    top=20mm
}
\hypersetup{    
    colorlinks=true,    
    linkcolor=black,    
    urlcolor=blue,    
    pdftitle={Storia}
}

% === COMMANDS ===
\newcommand{\figbox}[1]{ 
    \begin{figure*}[h!]        
        \begin{center}            
            \fbox{#1}        
        \end{center}    
    \end{figure*}
}

\newcommand*\circled[1]{
    \tikz[baseline=(char.base)]{            
        \node[shape=circle,draw,inner sep=1.1pt] (char) {#1};
    }
}

\newcommand\hr{\vspace{0.1cm}\par\vspace{-.5\ht\strutbox}\noindent\hrulefill\par\vspace{0.1cm}}

% Fill the remaining space of a wrapfigure
\newcommand{\wrapfill}{
    \par
    \ifnum \value{WF@wrappedlines} > 0
        \addtocounter{WF@wrappedlines}{-1}%
        \null\vspace{
            \arabic{WF@wrappedlines}
            \baselineskip
        }
        \WFclear
    \fi
    \phantom{}
}

% === TEXT ===

\title{\textbf{Storia\\Passerella 23-24}}
\author{Matteo Frongillo}

\begin{document}

\maketitle
\tableofcontents

\newpage
\part{Restaurazione e moti rivoluzionari}

\section{Restaurazione (1815-1848)}
La Restaurazione è il periodo storico dopo la sconfitta di Napoleone, il quale inizia nel 1815
con il Congresso di Vienna e termina con le rivoluzioni del 1848.

\subsection{Principi della Restaurazione}
\begin{itemize}
    \item Legittimità tradizionale, ossia il ritorno delle dinastie legittime al potere;
    \item Principio monarchico, che riafferma il potere monarchico.
\end{itemize}

\subsection{Storia della Restaurazione}
\begin{itemize}
    \item La restaurazione mirava a ristabilire l'ordine pre-rivoluzionario, trasformando i
        cittadini in sudditi obbedienti;
    \item Volontà dei popoli:
        \begin{itemize}
            \item Periodo caratterizzato da movimenti nazionalisti e liberali che si opponevano
                alla restaurazione;
            \item Richiesta di maggiori diritti civili e nazionali;
        \end{itemize}
    \item Santa Alleanza:
        \begin{itemize}
            \item Coalizione formata da Austria, Prussia e Russia (successivamente anche la
                Francia)
            \item Obiettivo comune di mantenere lo status quo e prevenire rivoluzioni;
            \item L'Inghilterra non era membro ufficiale della Santa Alleanza, ma collaborava
                per mantenere l'equilibrio di potere;
        \end{itemize}
    \item Movimenti politici:
        \begin{itemize}
            \item \textbf{Movimento liberale:} promuoveva le libertà individuali, costituzioni
                e governi rappresentativi;
            \item \textbf{Movimento conservatore:} sosteneva la conservazione delle monarchie
                e delle strutture sociali tradizionali.
        \end{itemize}
\end{itemize}

\section{Liberalismo}
\subsection{Origine del liberalismo}
\begin{itemize}
    \item Si forma in età moderna come dottrina politica contro l'assolutismo monarchico;
    \item Si oppone al giogo delle autorità.
\end{itemize}

\subsection{``Libertà da''}
\begin{enumerate}
    \item Più lo stato è limitato, più l'uomo è libero di agire (in contrapposizione con
        l'assolutismo);
    \item Libertà da uno Stato oppressore (contro l'assolutismo):
        \begin{itemize}
            \item Liberali conservatori o moderati;
            \item Libertà negativa, che si riferisce alla protezione dell'individuo
                dall'interferenza o coercizione dello Stato o di altri individui;
            \item Indotta dalla richiesta di un minor intervento dello Stato nella vita delle
                persone;
            \item Libertà moderata con libertà di voto solo per possidenti;
            \item Realizzazione graduale del liberalismo, con la possibilità di scendere a 
                compromessi con l'Antico Regime; 
        \end{itemize}
    \item Aspirazioni della ricca borghesia;
    \item Divisione dei poteri (Monarchia costituzionale):
        \begin{itemize}
            \item Parlamento eletto a suffragio censitario;
            \item Sovranità nazionale;
            \item Potere dello Stato limitato;
            \item Stato favorente della libertà d'azione.
        \end{itemize}
\end{enumerate}

\subsection{``Libertà di''}
\begin{enumerate}
    \item L'uomo è tanto più libero se può esercitare le proprie libertà, se non è limitato o
        escluso dalla povertà o dalla malattia;
    \item Chi non è limitato ha i mezzi per poter godere delle proprie libertà;
    \item Trasformazione rapida della società con cambiamenti decisi e con la volontà di
        stravolgere la società dell'Antico Regime;
    \item Volontà di coinvolgere tutta la popolazione (democratici) tramite referendum e
        suffragio universale; 
    \item Libertà di parola, riunione e associazione:
        \begin{itemize}
            \item Partecipare, ossia essere inclusi nelle decisioni politiche e civili;
            \item Diritti individuali:
                \begin{itemize}
                    \item Libertà di parola, riunione, associazione;
                    \item Libertà di stampa, culto, attività economica;
                \end{itemize}
        \end{itemize}
    \item Ideologie associate:
        \begin{itemize}
            \item Liberali conservatori/moderati: sostengono un equilibrio tra libertà e ordine;
            \item Liberali radicali/democratici: promuovono cambiamenti più profondi e
                partecipazione diretta.
        \end{itemize}
    \item Libera circolazione delle merci:
        \begin{itemize}
            \item ``Laissez-faire, Laissez-passer'': transizioni libere tra gruppi privati;
            \item Nessun intervento dello Stato nell'economia.
        \end{itemize}
\end{enumerate}

\subsection{Liberalismo in breve}
\begin{figure*}[ht!]
    \vspace*{.5cm}
    \begin{center}
        \begin{tikzpicture}
            % Draw the central ellipse
            \draw[thick] (0,0) ellipse (3 and 1.5);
            
            % orizzontali
            \draw[->, thick] (-6,0) -- (-3,0);
            \draw[->, thick] (3,0) -- (6,0);
        
            %oblique
            \draw[->, thick] (-3,0.7) -- (-2.3,1.3);
            \draw[->, thick] (-3,-0.7) -- (-2.3,-1.3);
            \draw[->, thick] (2.3,1.3) -- (3,0.7);
            \draw[->, thick] (2.3,-1.3) -- (3,-0.7);
        
            %verticali
            \draw[->, thick] (-4.5, 0) -- (-4.5, -1.3);
            \draw[->, thick] (0, 1.5) -- (0, 2.5);
            \draw[->, thick] (0, -1.5) -- (0, -2.5);
            
            % Labels
            \node at (-4.5, 0.3) {liberalismo};
            \node at (-4.5, -2.9) {\makecell[c]{libera circolazione\\delle merci\\``laissez-faire''\\``laissez-aller''\\ \textdownarrow \\ no interventi dello\\stato nell'economia}};
            \node at (0, -4.5) {\makecell[c]{libertà di\\partecipare\\ \textdownarrow\\liberali radicali,\\democratici\\ \textdownarrow\\ universatilità,\\uguaglianza\\politica}};
            \node at (0, 3.4) {\makecell[c]{liberali conservatori\\o moderati\\ \textuparrow\\ libertà da}};
            
            \end{tikzpicture}
    \end{center}
\end{figure*}

\newpage
\section{Mappa delle ideologie conservatrici e liberali}

\begin{figure*}[ht!]
    \vspace*{1cm}
    \begin{center}
            \begin{tikzpicture}
                \draw[->, thick] (0,0) -- (-11.5,0);

                \draw[thick] (-1,-0.15) -- (-1,0.15);
                \node at (-1,0.5) {Reazionari};
                \node at (-1,-0.9) {\makecell[c]{restaurare\\ciò che\\c'era prima}};

                \draw[thick] (-4,-0.15) -- (-4,0.15);
                \node at (-4,0.5) {Conservatori};
                \node at (-4,-1.1) {\makecell[c]{mutare il\\presente nel\\rispetto delle\\tradizioni}};

                \draw[thick] (-7,-0.15) -- (-7,0.15);
                \node at (-7,0.5) {Progressisti};
                \node at (-7,-1.1) {\makecell[c]{cambiamenti\\graduali,\\tramite\\riforme}};

                \draw[thick] (-10,-0.15) -- (-10,0.15);
                \node at (-10,0.5) {Rivoluzionari};
                \node at (-10,-1.3) {\makecell[c]{cambiamenti\\radicali\\e forzati,\\uso della\\violenza}};
            \end{tikzpicture}
    \end{center}
\end{figure*}

\begin{table*}[ht!]
    \begin{center}
        \begin{tabular}{| m{5cm} | m{5cm} |}
            \hline \; \textbf{Conservatori} & \; \textbf{Liberali} \\
            \hline - Monarchia assoluta; & - Monarchia costituzionale; \\
            \hline - Rafforzamento delle gerarchie & - Rispetto dell'autonomia \\
                    \; di potere (clero e nobilità); & \; dell'individuo e della ragione; \\
            \hline - Difesa della tradizione; & - Idea di progresso; \\
            \hline - Potere per diritto divino, & - Potere delle élite \\
                    \; nessuna rappresentanza & \; borghesi, che devono essere \\
                    \; popolare. & \; rappresentate. \\
            \hline
        \end{tabular}
    \end{center}
\end{table*}

\section{Socialismo}
\begin{itemize}
    \item Ruolo dello Stato nell'economia;
    \item Meno individualismo;
    \item No liberalismo economico;
    \item Diritti dei lavoratori;
    \item Abolizione della proprietà privata;
    \item Redistribuzione della ricchezza;
    \item Proletari vs. borghesia 
\end{itemize}

\subsection{Assieme alla rivoluzione industriale}
\begin{itemize}
    \item Mutano in modo drastico milioni di vite;
    \item Radicale cambiamento nel sistema di produzione;
    \item Massiccio impiego di donne e bambini nei lavori;
    \item Nelle città sorgono quartieri operai privi di servizi elementari;
    \item Trasformazione dal mondo agricolo a quello industriale, con conseguente mutamento
        della società e la creazione di nuove classi sociali.
    \item Trade unions: si creano nuovi sindacati rappresentativi delle classi operaie;
    \item Leggi a tutela dei lavoratori.
\end{itemize}

\section{Socialismo scientifico}
\subsection{Prima idea - Contro il capitalismo}
\begin{itemize}
    \item Il socialismo sfutta le contraddizioni e il fallimento del capitalismo per crescere;
    \item Scontro tra borghesia e proletariato;
    \item Attuazione del socialismo e poi del comunismo:
        \begin{itemize}
            \item Socialismo: si riceve in base al proprio lavoro;
            \item Comunismo: si riceve in base alla propria necessità.
        \end{itemize}
\end{itemize}

\subsection{Seconda idea - Lotta tra classi}
\begin{itemize}
    \item La società umana è passata attraverso 4 fasi caratterizzate dalle lotte tra gli
        oppressi e gli oppressori:
        \begin{enumerate}
            \item Communità primitive;
            \item Regime di schiavitù;
            \item Società feudale;
            \item Società borghese;
        \end{enumerate} 
    \item Scontro tra oppressi e oppressione:
        \begin{itemize}
            \item Maturazione delle condizioni strutturali;
            \item Cambiamento nei mezzi di produzione della ricchezza.
        \end{itemize}
\end{itemize}

\subsection{Terza idea - Modi di produzione della ricchezza}
\begin{itemize}
    \item Concettro di struttura e sovrastruttura:
        \begin{itemize}
            \item Struttura: rapporti economici;
            \item Sovrastruttura: tutto il resto (cultura, Stato, religione, politica);
            \item Gli uomini e la loro organizzazione dipendono dal modo in cui producono ciò
                di cui necessitano;
            \item Attività condizionata dai mezzi di produzione di cui gli uomini dispongono
        \end{itemize}
    \item 4 fasi della rivoluzione comunista:
        \begin{enumerate}
            \item Rivoluzione della classe operaia;
            \item Dittatura del proletariato;
            \item Società socialista;
            \item Società comunista:
                \begin{itemize}
                    \item Abolizione della proprietà privata;
                    \item Creazione di una società priva di classi e distinzioni sociali.
                \end{itemize}
        \end{enumerate}
\end{itemize}

\section{Concetti del socialismo}
\subsection{Concetto naturalistico}
\begin{itemize}
    \item Fattore etnici:
        \begin{itemize}
            \item Profilo etnico o consanguineità;
            \item Legame con il territorio;
            \item Lingua;
            \item Cultura;
            \item Consuetudini;
            \item Esperienze storiche;
            \item Tradizioni civili e religiose.
        \end{itemize}
    \item Stirpe originale;
    \item Recupero dell'identità originaria della nazione;
    \item Non c'è una scelta volontaria del popolo, poiché la nascita dell'idea precede la
        costituzione di uno stato e l'appartenenza nazionale di una persona è predestinata;
    \item Romanticismo e Restaurazione ne sono il contesto storico (concezione tedesca).
\end{itemize}

\subsection{Concetto volontaristico}
\begin{itemize}
    \item Prodotto culturale: stare assieme sulla base di esperienze comuni;
    \item Caratteristiche legate alla volontà di condividere cultura e storia;
    \item Eredità di ricordi e consenso attuale:
        \begin{itemize}
            \item Eredi di un patrimonio comune;
            \item La nazione esiste finché trova posto nella mente e nel cuore delle persone;
        \end{itemize}
    \item È una scelta.
    \item Illuminismo e Riv. Francese come radici culturali di questo concetto (concezione
        francese).
\end{itemize}

\newpage
\part{La Svizzera della vecchia confedazione al 1848}
\section{Differenza tra i tipi di Stato}

\textbf{Confederazione:} Alleanza tra Stati autonomi

\textbf{Federazione:} Unico Stato composto da più parti che godono di una larga autonomia

\textbf{Repubblica:} Unico Paese composto da più Stati confederati

\section{Ancien Régime (fino al 1798)}

\subsection{Confederazione dei 13 Cantoni}

La Confederazione dei 13 Cantoni era basata su:

\subsubsection{Tre gruppi territoriali}
\begin{enumerate}
    \item Cantoni sovrani (13);
    \item Territori soggetti (baliaggi):
        \begin{itemize}
            \item Per amministrare il baliaggio, i cantori sovrani mandavano un responsabile
                chiamato Landfogto;
        \end{itemize}
    \item 11 alleati;
\end{enumerate}

\subsubsection{Ampia autonomia delle comunità locali}
(Entità indipendente del Sacro Romano Impero Germanico [SRIG] dal 1648).

Esistevano tre tipi di governo:
\begin{enumerate}
    \item Governi oligarchici;
    \item “Landsgemeinde” = Assemblee di contadini proprietari;
    \item Dieta = Assemblea dei delegati dei 13 Cantoni e di alcuni paesi alleati;
\end{enumerate}

Ultima Dieta nel 1798

\subsubsection{Mercenariato}
I soldati svizzeri venivano pagati per prestare servizio in altri eserciti.
\begin{itemize}
    \item Club Elvetico in Francia → Intellettuali svizzeri che vivono in Francia per
        amministrare come ambasciata estera il rapporto tra Francia e Svizzera.
\end{itemize}

\subsection{Riassunto}
\begin{itemize}
    \item Tipo di Stato: Confederazione dei 13 Cantoni e 11 alleati;
    \item Legge fondamentale: Patti tra i cantoni. I cantoni sono sovrani (Confederazione);
    \item Potere centrale: Dieta (ev. anche Club Elvetici);
    \item Confederazione: La Svizzera è come un grappolo d’uva, tutti gli acini sono separati
        l’uno dall’altro ma tenuti assieme dallo stesso ramo;
\end{itemize}

\newpage
\section{Repubblica Elvetica (1798 – 1802)}
La Svizzera venne invasa e conquistata dalla Repubblica francese di Napoleone Bonaparte,
spronata anche dai Club Elvetici, i quali aderivano fortemente al potere napoleonico.

La Svizzera viene conquistata dai francesi nel 1798 e diventa una Repubblica con una
costituzione fornita dalla Francia.

\subsection{Dalla Costituzione della Repubblica Elvetica, 1798}
\begin{itemize}
    \item La Svizzera è una e indivisibile;
    \item È un Regime imposto dalla Francia;
    \item Forte potere centrale;
    \item Divisione territoriale (suddiviso in Cantoni);
    \item Ordinamento istituzionale;
    \item Abolizione dei baliaggi → i Cantoni sono unità amministrative.
\end{itemize}

\subsection{Riforme introdotte con la Repubblica}
\begin{itemize}
    \item Uguaglianza dei cittadini di fronte alla legge;
    \item Suffragio universale maschile;
    \item Libertà di pensiero, di stampa e di religione;
    \item Libertà di domicilio e industria;
    \item Cittadinanza svizzera: Le persone non sono più cittadine unicamente nei loro cantoni;
    \item Divisione dei poteri secondo il principio di Montesquieu;
    \item Unificazione dei paesi, delle misure e della moneta;
    \item Soppressione delle dogane interne;
    \item Obbligatorietà dell’insegnamento elementare.
\end{itemize}

\subsection{Riassunto}
\begin{itemize}
    \item Tipo di stato: Stato Unitario, Repubblica una e indivisibile;
    \item Legge fondamentale: Tutte le riforme citate;
    \item Potere centrale: Esiste un Potere centrale con tutte le competenze per governare:
    \begin{itemize}
        \item Esecutivo: Direttorio (5 membri);
        \item Legislativo: Senato (4 membri/cant.) e Gran Consiglio (8 membri/cant.);
        \item Giudiziario: Tribunale elvetico (19 membri, 1 per cantone);
    \end{itemize}
    \item Repubblica: La Svizzera è come una mela, compatta e unificata. Il torsolo è il potere
        centrale.
\end{itemize}

\newpage
\section{La Costituzione della Mediazione (1802)}
Napoleone ritira tutte le truppe dalla Svizzera, ciò porta a conflitti civili nei cantoni.
Dopo due mesi di guerre, le truppe napoleoniche tornano nella Repubblica elvetica per mediare
gli scontri interni.

\subsection{Suddivisioni interne}

\subsubsection{Federalisti (cattolici)}
Cantoni attaccati alle tradizioni cattoliche;
\begin{itemize}
    \item Ampia autonomia cantonale;
    \item Sovranità cantonale.
\end{itemize}

\subsubsection{Repubblicani (centralisti)}
Cantoni attaccati a una Svizzera unita e con un potere centrale unico;
\begin{itemize}
    \item Governo indivisibile;
    \item Cantoni amministrativi senza autonomia.
\end{itemize}

\subsection{Atto di Mediazione (1803 – 1810)}
L’atto di Mediazione è stato ideato a Parigi nel 1803 e sarebbe stato un intervento per
pacificare le fazioni svizzere in conflitto.

\subsubsection{Obiettivi}
\begin{itemize}
    \item Ristabilire la pace e l’ordine in Svizzera;
    \item Conciliare le diverse fazioni politiche;
    \item Preservare l’indipendenza e l’integrità della Svizzera.
\end{itemize}

\subsubsection{Principali provvedimenti}
\begin{itemize}
    \item Libertà fondamentali vengono garantite (Repubblicano);
    \item Ogni Cantone ha la propria costituzione (Federalista);
    \item Ogni Cantone ha un’ampia autonomia (Repubblicano);
    \item Una parte del potere è amministrato da un’autorità federativa (Federalista);
    \item Dieta ridotta.
\end{itemize}

\subsubsection{Ruolo della Dieta ridotta}
N.B.: La Dieta NON è un parlamento, ma un’autorità centrale (Congresso dei delegati)
\begin{itemize}
    \item Può dichiarare guerra;
    \item Può promulgare decreti con il 75\% dei voti favorevoli;
    \item Regolare conflitti tra cantoni.
\end{itemize}

\subsubsection{Ruolo del Cantone Ticino}
\begin{itemize}
    \item Partecipare al Potere Legislativo: Gran Consiglio (110 deputati);
    \item Partecipare al Potere Esecutivo: Piccolo Consiglio (9 deputati).
\end{itemize}

\subsection{Riassunto}
\begin{itemize}
    \item Tipo di stato: Stato Federale con 19 cantoni che hanno libertà nel gestire al meglio
        la loro costituzione, il loro territorio e la loro indipendenza;
    \item Legge fondamentale: Atto di Mediazione, libertà fondamentali, uguaglianza dei
        cittadini di fronte alle leggi, soppressione delle dogane interne;
    \item Potere centrale: Autorità centrale → Dieta ridotta;
    \item Federazione: La Svizzera è come un mandarino, gli spicchi (Cantoni) sono divisi tra
        loro ma sostenuti assieme dalla buccia (Autorità centrale/Dieta ridotta).
\end{itemize}

\newpage
\section{La Restaurazione e la Rigenerazione}

\subsection{Restaurazione: Patto del 1815}

\subsubsection{Congresso di Vienna (1815)}
\begin{itemize}
    \item Ridisegno della cartina europea;
    \item Viene riconosciuta a livello internazionale la neutralità elvetica perpetua:
        \begin{itemize}
            \item Durante l’invasione di Napoleone la Svizzera è stata costretta ad allearsi con
                la Francia, dunque non è stata sempre neutrale.
        \end{itemize}
    \item Confederazione:
    \begin{itemize}
        \item Dieta (75\% dei voti) → Politica estera:
            \begin{itemize}
                \item La Dieta rimarrà sempre un’assemblea dei delegati;
            \end{itemize}
        \item Esercito comune.
    \end{itemize}
\end{itemize}

\subsection{Rigenerazione (1831 – 1848)}
\begin{itemize}
    \item Nuove costituzioni più liberali per una decina di cantoni.
\end{itemize}

\subsubsection{Cantoni radicali (liberali)}
\begin{itemize}
    \item Revisione del Patto del 1815;
    \item Più accentramento;
    \item Stato e società laici.
\end{itemize}

\subsubsection{Cantoni conservatori (leghisti)}
\begin{itemize}
    \item Unione di difesa:
        \begin{itemize}
            \item Guerra del Sonderbund;
        \end{itemize}
    \item Forte autonomia centrale;
    \item Difesa del cattolicesimo.
\end{itemize}

\subsubsection{Guerra del Sonderbund}
Il conflitto del Sonderbund avvenne a causa della Dieta che votò per lo scioglimento della Lega
(1847) poiché la riteneva anticostituzionale.\\
I militi leghisti si unirono per combattere e preservare la propria ideologia.\\
La guerra fu tra l’Esercito confederale condotto da Dufour e i militi del Sonderbund.

\subsubsection{Riassunto}
\begin{itemize}
    \item Tipo di stato: Confederazione di 22 Cantoni sovrani;
    \item Legge fondamentale Restaurazione: Congresso di Vienna e Patto del 1815, i Cantoni 
        tornano sovrani. Viene imposta una certa censura e viene ristretta la libertà generale
        per paura delle innovazioni portate dalla Rivoluzione Francese;
    \item Legge fondamentale Rigenerazione: Una decina di Cantoni promulga nuove costituzioni
        più liberali;
    \item Potere centrale: Dieta con poteri limitati, Esercito comune;
    \item Confederazione: La Svizzera è come un grappolo d’uva, tutti gli acini sono separati
        l’uno dall’altro ma tenuti assieme dallo stesso ramo.
\end{itemize}

\newpage
\section{Svizzera Moderna (1848 – oggi)}

\subsection{La Costituzione del 1848}
La Svizzera torna una Federazione:
\begin{itemize}
    \item Nel 1848 la Dieta accetta la nuova Costituzione;
    \item Referendum nei cantoni: 15 cantoni + 1 semi-cantone a favore;
    \item 12 Settembre 1848: La Svizzera ritorna uno Stato federale.
\end{itemize}

\subsection{Poteri suddivisi}
\begin{itemize}
    \item Potere legislativo: Assemblea federale:
        \begin{itemize}
            \item I cittadini eleggono:
            \begin{itemize}
                \item Consiglio nazionale: In proporzione al numero di abitanti;
                \item Consiglio degli Stati: 2 rappresentanti per Cantone;
            \end{itemize}
        \end{itemize}
    \item Potere esecutivo: Consiglio Federale (7 membri):
        \begin{itemize}
            \item Vengono eletti dal Potere legislativo;
        \end{itemize}
    \item Potere giudiziario: Tribunale Federale:
        \begin{itemize}
            \item Vengono eletti dal Potere legislativo.
        \end{itemize}
\end{itemize}

\subsection{La democrazia semi-diretta}
\begin{itemize}
    \item Sovranità popolare per mezzo di rappresentanti eletti (intermediari);
    \item Referendum e Iniziativa.
\end{itemize}

\subsubsection{Introduzione al diritto di referendum (1874)}
Facoltativo. Con 30'000 firme (50'000 dal 1977) è possibile sottoporre al voto del popolo una
legge già accettata dal parlamento federale.

\subsubsection{Introduzione al diritto di iniziativa costituzionale (1891)}
Con una richiesta approvata da 50'000 svizzeri (100'000 dal 1977) i cittadini possono proporre
l’introduzione, l’abrogazione o la modifica di un articolo della costituzione.

\subsection{Composizione politica}
\begin{itemize}
    \item Politica su tre livelli: comunale, cantonale e confederale;
    \item A livello federale divisione dei poteri:
        \begin{itemize}
            \item Esecutivo: Consiglio federale formato da 7 membri eletti ogni 4 anni;
            \item Legislativo bicamerale: Consiglio nazionale e Consiglio degli Stati;
            \item Giudiziario: Tribunale federale.
        \end{itemize}
\end{itemize}

\subsection{I movimenti politici della Svizzera}

\subsubsection{Tendenza liberale-radicale}
\begin{itemize}
    \item Favorevole al rafforzamento dello Stato Federale all’affermazione dei diritti
        individuali e politici;
    \item 1894 → Partito radical-democratico.
\end{itemize}

\subsubsection{Tendenza conservatrice}
\begin{itemize}
    \item Difende l’autonomia dei cantoni e la libertà delle chiese.
\end{itemize}

\subsubsection{Cattolici conservatori}
\begin{itemize}
    \item Mirano alla riconciliazione tra Chiesa e Stato;
    \item 1894 → Partito Popolare Cattolico.
\end{itemize}

\subsubsection{Partito Socialdemocratico svizzero}
\begin{itemize}
    \item Risponde ai problemi sorti dall’industrializzazione del 1888.
\end{itemize}

\subsection{La formula magica}
\begin{itemize}
    \item Per alcuni decenni: tutte le poltrone dei governi in mano ai liberal-radicali;
    \item 1891: Entra per la prima volta un cattolico-conservatore (Joseph Zemp);
    \item 1959-2003: La composizione partitica del Consiglio Federale è rimasta immutata
        (formula magica):
        \begin{itemize}
            \item 2 liberali;
            \item 2 PPD;
            \item 2 socialisti;
            \item 1 UDC.
        \end{itemize}
\end{itemize}

\subsection{Riassunto}
\begin{itemize}
    \item Tipo di stato: Stato federale (chiamato per motivi pratici Confederazione) di 26
        Cantoni. Regime di democrazia semidiretta;
    \item Legge fondamentale: Costituzione federale. Potere ripartito tra cantoni e
        confederazione. Cantoni sovrani fin dove la loro sovranità non è limitata dalla
        Costituzione federale. Le costituzioni dei cantoni sono repubblicane e democratiche;
    \item Potere centrale: Federazione suddiviso in Potere esecutivo, legislativo e giudiziario;
    \item Federazione: La Svizzera è come un mandarino, gli spicchi (Cantoni) sono divisi tra
        loro ma sostenuti assieme dalla buccia (Autorità centrale).
\end{itemize}

\newpage
\part{L'età dell'Imperialismo}
\section{Definizione di Imperialismo}
``Momento volto alla costituzione di Imperi coloniali da parte delle potenze industriali europee
con lo scopo di procurarsi materie prime necessarie all’industria ed esportarvi prodotti finiti''.

\section{Imperialismo}
\subsection{Paesi sovrani}

\subsubsection{Dal 1830}
\begin{itemize}
    \item Stati Uniti d’America;
    \item Giappone.
\end{itemize}

\subsubsection{Dal 1870 – 1914 (1945)}
\begin{itemize}
    \item Francia;
    \item Gran Bretagna;
    \item Germania;
    \item Belgio;
    \item Olanda;
    \item Italia.
\end{itemize}

\subsection{Dottrina Monroe (1823)}
``L’America agli Americani''.
\begin{itemize}
    \item Problema: Gli USA non tollerano nessuna intromissione nell’emisfero occidentale da
        parte delle potenze europee;
    \item Soluzione: I paesi europei puntano alla colonizzazione del continente africano.
\end{itemize}

\subsection{Grande Depressione (1873)}

\subsubsection{Problema della sovrapproduzione}
\begin{itemize}
    \item Problema: I materiali prodotti non trovano più acquirenti nei paesi di produzione;
    \item Soluzione: Vendita dei prodotti e investimento dei capitali alle colonie.
\end{itemize}

\subsection{Conferenza di Berlino (1884 / 1885)}
\begin{itemize}
    \item Obbligo di notifica da parte di uno Stato verso gli altri Stati in caso di occupazione
        di un territorio nel continente africano;
    \item Il Congo, conteso da molti Stati, viene donato all’Imperatore belga Re Leopoldo II:
        \begin{itemize}
            \item Diventa Stato Indipendente del Congo (1885).
        \end{itemize}
\end{itemize}

\newpage
\section{La storia del Congo}

\subsection{Conseguenze della colonizzazione congolese}

\subsubsection{Conferenza di Berlino (1885)}
\begin{itemize}
    \item Il Congo viene donato all’Imperatore belga Re Leopoldo II
    \item Il Congo diventa Stato Indipendente del Congo
\end{itemize}

\subsection{Arricchimento dal Congo}
\begin{itemize}
    \item Caucciù → Gomma → Pneumatici
    \item Avorio
\end{itemize}

\subsection{Creazione della “Force publique” (Polizia autoctona)}
\begin{itemize}
    \item Presa in ostaggio di chi non è abile al lavoro
    \begin{itemize}
        \item Gli idonei dovevano fornire un quantitativo di gomma per salvare gli ostaggi
    \end{itemize}
    \item La violenza non giustificata da parte della F.P. veniva punita
    \begin{itemize}
        \item Per verificare che i proiettili sparati fossero stati usati contro una rivolta, gli agenti della F.P. dovevano mutilare e consegnare la mano destra del bersaglio
    \end{itemize}
\end{itemize}

\subsection{Discorso al parlamento francese (1885)}
“Discorso al parlamento francese” di Jules Ferry

\section{Le cause (e motivazioni) dell’Imperialismo}

\subsection{Fattori economici}
\begin{itemize}
    \item Ricerca di materie prime e di mercati (IIa rivoluzione industriale)
    \item Sbarchi commerciali
    \item Espansione del capitalismo
    \item Investimenti
\end{itemize}

\subsection{Fattori culturali e ideologici}
\begin{itemize}
    \item Aspetto umanitario e civilizzatore
    \item “Razze superiori” che hanno il compito e il dovere di civilizzare quelle “inferiori”
\end{itemize}

\subsection{Fattori politici (nazionalismo e militarismo)}
\begin{itemize}
    \item Gli Stati tra loro sono rivali
    \begin{itemize}
        \item Per non soccombere è necessario attuare una politica di potenza
        \item La politica coloniale è un mezzo per dimostrare la propria potenza e la propria supremazia come Stato
    \end{itemize}
\end{itemize}

\section{Insegnare il concetto di Nazione}

\subsection{Rafforzamento del proprio Stato}
\begin{itemize}
    \item Stato forte
    \begin{itemize}
        \item Rafforzamento dell’esecutivo
        \item Politica estera più aggressiva
        \item Rafforzamento del nazionalismo
        \begin{itemize}
            \item Superiorità nazionale
            \item Esecutivismo nazionale
            \item Dovere di insegnare la Nazione allo Stato debole
        \end{itemize}
    \end{itemize}
\end{itemize}

\subsection{Metodo di insegnamento tramite sistemi politici}
\begin{itemize}
    \item Discorso nazionalistico che dà una base di legittimazione per le élite politiche
    \begin{itemize}
        \item Scuola
        \item Esercito
        \item Rituali pubblici
    \end{itemize}
\end{itemize}

\subsection{Scolarizzazione}
\begin{itemize}
    \item Sistemi educativi elementari obbligatori
    \item Storia e letteratura nazionale
    \begin{itemize}
        \item Sentimenti patriottici → esaltazione ad essere eroi nazionali
    \end{itemize}
\end{itemize}

\subsection{Esercito}
\begin{itemize}
    \item Nazionalizzazione delle masse
    \begin{itemize}
        \item Servizio militare obbligatorio per la patria, dunque anche per lo Stato debole poiché fa parte della patria
    \end{itemize}
\end{itemize}

\subsection{Rituali pubblici}
\begin{itemize}
    \item Bandiera nazionale
    \item Inni nazionali
    \item Feste nazionali
\end{itemize}

\subsection{Movimenti nazionalisti}
\begin{itemize}
    \item Esasperazione del sentimento di comune appartenenza
    \item Superiorità della propria nazione
    \item Politica di potenza per sottolineare la propria superiorità
    \item IMPERIALISMO
\end{itemize}

\section{Il razzismo scientifico}

\subsection{Illuminismo (1694)}

\subsubsection{Pensiero razionale}
\begin{itemize}
    \item Pone l’uomo al centro dell’universo
    \item L’uomo viene classificato secondo due macrocategorie:
    \begin{itemize}
        \item Natura
        \item Classici
    \end{itemize}
\end{itemize}

\subsubsection{Scienza $\leftrightarrow$ Estetica}
\begin{itemize}
    \item Scienza: Classificazione delle razze in base al loro posto nel mondo
    \item Estetica: È manifestata in modo tangibile, dunque misurabile con l’appagamento dell’occhio verso gli individui
\end{itemize}

\subsection{Darwinismo sociale}

\subsubsection{Teorie razziste}
\begin{itemize}
    \item Applicazione allo studio delle società umane i principi darwiniani della lotta per l’esistenza e della selezione naturale
    \item Ideologia razzista
    \begin{itemize}
        \item Superiorità culturale $\leftrightarrow$ Superiorità biologica
    \end{itemize}
\end{itemize}

\subsubsection{Passaggio dalla teoria razziale al razzismo (J. A. Gobineau)}

\subsubsection{Classificazione delle razze}
\begin{itemize}
    \item Al mondo esistono 3 razze:
    \begin{itemize}
        \item Bianca → Superiore (Razza che deve essere preservata)
        \item Gialla → Inferiore
        \item Nera → Inferiore
    \end{itemize}
    \item Preservazione della razza dominante
    \begin{itemize}
        \item Gli ariani non si devono mischiare con altre razze
        \item Rischio di DEGENERAZIONE
        \item Eugenetica → Sterilizzazione chimica delle persone “anormali”
    \end{itemize}
\end{itemize}

\subsubsection{Popolazioni ebraiche}
\begin{itemize}
    \item Sghettizzazione → Ghetti vengono smantellati e gli ebrei si reintegrano nella popolazione
\end{itemize}

\subsubsection{Zoo umani → Esposizioni universali (expo)}
\begin{itemize}
    \item Padiglioni che ricreavano gli “habitat” delle diverse popolazioni e razze
\end{itemize}













\end{document}
