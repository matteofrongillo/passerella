\documentclass{article}
\usepackage{frongillo}

\title{\textbf{Sintesi di italiano\\Corso Passerella}}
\author{Matteo Frongillo}

\begin{document}

\maketitle
\tableofcontents
\newpage

\section{Dante}
\begin{itemize}
    \item Nasce a Firenze nel 1265;
    \item Si innamora di Beatrice ma lei venendolo a sapere li toglie il saluto:\\
        Nonostante ciò, Dante continua a dedicarle poesie: questo rappresenta la forma più
        pura di amore;
    \item Beatrice muore nel 1290.
\end{itemize}

\subsection{Vita nova}
\begin{itemize}
    \item È il suo primo prosimetro;
    \item Raccolta di poesie dedicate a Beatrice, pubblicato in suo onore dopo la sua morte;
    \item Non c'è nessun'indicazione di tempo o di luogo per tutta la raccolta.
\end{itemize}

\subsubsection{Oltre la spera}
\poetry{
    & \textbf{Dante, Vita nova. XLI 10-13}\\\\
    & \textit{Oltre la spera che più larga gira}\\
    & \textit{passa ’l sospiro ch’esce del mio core:}\\
    & \textit{intelligenza nova, che l’Amore}\\
    4. & \textit{piangendo mette in lui, pur su lo tira.}\\\\
    & \textit{Quand’elli è giunto là dove disira,}\\
    & \textit{vede una donna, che riceve onore,}\\
    & \textit{e luce sì, che per lo suo splendore}\\
    8. & \textit{lo peregrino spirito la mira.}\\\\
    & \textit{Vedela tal, che quando ’l mi ridice,}\\
    & \textit{io no lo intendo, sì parla sottile}\\
    11. & \textit{al cor dolente, che lo fa parlare.}\\\\
    & \textit{So io che parla di quella gentile,}\\
    & \textit{però che spesso ricorda Beatrice,}\\
    14. & \textit{sì ch’io lo ’ntendo ben, donne mie care.}
}

\textbf{Prima quartina}
\begin{enumerate}
    \item Quartina bipartita a metà
    \item Nella prima metà: il sospiro di Dante si stacca e va verso il Paradiso (indicazione di
        luogo)
    \item Sistema geocentrico: l'uomo e la Terra al centro dell'universo (poiché è una creazione
        divina)
    \item Oltre la spera che più larga gira: perifrasi che indica l'Empireo
    \item Secondo verso: il sospiro che si stacca dal suo cuore e va in Paradiso attraversa
        tutti e 9 i cieli grazie ad un \underline{movimento verticale}
    \item Il sospiro fa un viaggio dal basso verso l'alto
    \item Il viaggio lo fa la sua anima poiché lui è ancora vivo
    \item Nella seconda metà: dice cosa ha permesso questo movimento verticale dell'anima 
    \item Intelligenza nuova è la capacità più raffinata di concepire le cose, derivata dalla
        sua esperienza amorosa (v.3)
    \item Questa nuova intelligenza lo tira verso l'alto (v.4) con costanza
    \item Amore (v.3) è personificato (stilnovo) rappresentato come una divinità guerriera
        armata che attacca il poeta
    \item Grazie all'esperienza amorosa dolorosa Dante fa passare il suo sospiro piangendo (v.4)
    \item Con il pianto raffina la sua sensibilità
    \item Ambivalenza intelligenza e piangendo: si sviluppa una nuova conoscenza ma a costo del
        dolore
\end{enumerate}

\textbf{Seconda quartina}
\begin{enumerate}
    \item Lo spirito indipendentemente giunge in Paradiso, che è dove voleva andare (v.1)
    \item Descrive quello che il sospiro vede una volta arrivato in Paradiso: una donna
    \item La donna è ornata da tutti (v.6), da Dio, la Madonna, dagli angeli, altri beati e
        anche dall'anima di Dante
    \item La donna brilla talmente tanto che lo spirito pellegrino la guarda, ammirandola
    \item Si possono individuare una serie di parole appartenenti al campo semantico della
        vista: ``vede, luce, splendore, mira'' così da definire il Paradiso dantesco, ossia
        un impero di pura luce
\end{enumerate}

\textbf{Prima terzina}
\begin{enumerate}
    \item La suddivisione è ripresa per anadiplosi (la mira $\Longleftrightarrow$ Vedela)
    \item Si parla del ritorno al cuore del sospiro di Dante (ascesa)
    \item Lo spirito una volta tornato gli riferisce ciò che ha visto TALE a come l'ha visto
    \item Dante non capisce cosa dice il sospiro poiché usa un registro aulico troppo sottile
        (v.10)
    \item È impossibile per il sospiro descrivere quello a cui ha assistito e che ha provato
        durante il viaggio con parole semplici
    \item Topos dell'ineffabilità: quando un'esperienza ci trascende, la lingua non può
        descrivere tutto
    \item Il sospiro parla anche al cuore di Dante (v.11)
    \item Dante e il cuore non vanno visti come due enti diversi, bensì come una persona unica
    \item Il cambio di ascoltatore dello spirito serve ad accentuare l'incomprensione delle sue
        parole
    \item "Vedela" è scritto così per la legge di Tobler-Mussafia, che prevede l'inversione tra
        il pronome atono (debole) e il verbo (forte)
    \item Dante capisce solo dopo che il cuore gli riferisce, sottoforma di sentimento, ciò che
        lo spirito ha visto e ha detto al cuore
\end{enumerate}

\textbf{Seconda terzina}
\begin{enumerate}
    \item Parla dell'unica cosa che Dante ha capito dalla descrizione, ossia che la donna
        beata vista in Paradiso dal suo sospiro è la sua donna (Beatrice)
    \item Dante è certo che l'anima lucente era quella di Beatrice
    \item ``So io che'' significa perlomeno, ``però'' significa perciò
    \item Descrive Beatrice con l'aggettivo ``gentile'', inteso come nobile, pura d'animo
    \item ``Donne mie care'' indica che le destinatarie sono tutte le donne incontrate
        precedentemente
\end{enumerate}

\textbf{Analisi}
\begin{enumerate}
    \item Il sonetto è dipartito:
        \begin{itemize}
            \item L'ascesa, in quartine, è legata da un campo semantico sulla vista e al viaggio
                intrapreso dall'anima di Dante $\rightarrow$ Possibilità di vedere per lo spirito
            \item La discesa, in terzine, è legata da un campo semantico sulla comprensione
                $\rightarrow$ impossibilità di parlare per lo spirito e di capire per Dante
        \end{itemize}
\end{enumerate}

\textbf{Prosa}
\begin{enumerate}
    \item Nella parte in prosa, Dante afferma di aver avuto una visione che gli ha impedito di
        parlare di Beatrice
    \item Fino a quando non avrebbe potuto parlarne in modo più degno, Dante non avrebbe potuto
        nominarla
    \item La nominerà fino a quando riuscirà a dire quello che mai è stato detto di una donna
    \item Dante enuncia che dopo la visione ha cambiato idea su Beatrice, decidendo di non
        parlarne più finché non lo farà in maniera più degna
    \item Dante dunque preannuncia di avere un progetto in cui lo farà (Commedia)
\end{enumerate}

\newpage
\section{Commedia}
\begin{enumerate}
    \item La Commedia è un poema: poesia narrativa
    \item Si compone in 3 cantiche (Inferno, Paradiso, Purgatorio) composte da 33 canti + 1 (100)
    \item La data del viaggio è la settimana santa
    \item Le sue guide saranno Virgilio fino al paradiso terrestre, Beatrice fino quasi al
        paradiso empireo e San Bernardo fino a Dio
    \item Concezione tolemaica: Terra al centro poiché una creazione di Dio sta al centro
    \item Terra composta da due emisferi, uno abitabile e uno fatto d'acqua
    \item Al centro delle Terre emerse c'era Gerusalemme
    \item A Gerusalemme c'è una porta per entrare nella via dell'Inferno
    \item Lucifero è un mostro (ex angelo) a tre teste che mastica i 3 più grandi peccatori
        secondo lui mai esistiti: Giuda (traditore di Cristo), Cassio e Bruto (di Cesare)
    \item Inferno suddiviso in:
        \begin{enumerate}
            \item Ignavi (non battezzati)
            \item Spiriti magni (coloro che non hanno mai conosciuto Cristo)
            \item Incontinenti (lussuriosi, golosi, prodighi e avari, iracondi e accidiosi)
            \item Eretici ed epicurei (coloro che andavano contro o che dubitavano del dogma
                della religione)
            \item Violenti (contro il prossimo, contro sé stessi, contro Dio, natura e arte)
            \item Fraudolenti (mezzani e seduttori, adulatori, simoniaci, indovini, barattieri,
                ipocriti, ladri, consiglieri fraudolenti, seminatori della discordia e falsari)
            \item Traditori (dei parenti, della patria, degli ospiti e dei benefattori)
            \item Lucifero (traditori di Cristo e dell'impero di Cesare)
        \end{enumerate}
    \item Le punizioni nell'Inferno sono legate alla legge del contrappasso per analogia,
        ossia avevano una punizione simile, oppure per contrario, ossia avevano una punizione
        contraria a quello che era il peccato più caratteristico del dannato
\end{enumerate}

\subsubsection{Inferno, Canto I}
\begin{enumerate}
    \item La vita media nel Medioevo, come dato Biblico, era di 70 anni.
    \item Dante aveva circa 35 anni, data di inizio del viaggio: 1300
    \item Dante sceglie l'anno del primo Giubileo della storia (purificazione dei peccati)
        durante la settimana santa del 1300
    \item ``Nosta vita'', Dante amplia la prospettiva su tutta l'umanità, rappresentando TUTTE
        le persone
    \item La selva oscura rappresenta un periodo di peccato di Dante, nella quale non entra la Luce Divina.
    \item Dante spiega come abbia smarrito la diritta via ossia la Luce, la via verso Dio. 
        La via era smarrita poiché smarrita significa che può essere ritrovata.
    \item I tre aggettivi della Selva: Selvaggia (Disumana, senza uomo), Aspra (Intricata), Forte (Difficile da uscire). Disposti in Climax ascendente.
    \item Dante compara la Selva alla morte, dicendo che la Selva era poco meno spaventosa della morte stessa.
    \item La selva era spaventosa, disumana e intricata ma Dante vuole parlare comunque di quello che trova al suo interno. 
        Al ``ben'' si riferisce a ciò che di buono Dante trova, dunque Virgilio. Al ``l'altre cose'' si riferisce a tutto ciò che non era buono, dunque le Fiere.
    \item {\color{red}{Con la prospettiva ampliata si può chiudere il capitolo della Vita nova, poiché la privatissima vita amorosa di Dante viene ampliata con tutti i conoscenti di Dio che incontrerà e con tutto lo spazio percorso}}
    \item Dante intende la verace via come la retta via.
    \item Il sonno intende la pigrizia, poiché abbandonando il bene la coscienza viene abbandonata.
    \item Dante ALZA GLI OCCHI, gesto simbolico che indica lui che guarda verso la salvezza.
    \item Selva e Sole sono due opposti: in basso l'oscurità, in alto la luce, in basso il peccato, in alto la salvezza.
    \item Similitudine: Così come un naufrago si volge a guardare il mare periglioso, Dante si volge a guardare la Selva spaventosa, dicendo che chi persevera nel peccato è destinato a morire.
    \item Dante riprende il viaggio, iniziando a camminare verso il pendio.
    \item Ed ecco: Espressione avversativa di Dante.
    \item La Lonza rappresenta la lussuria: rapida, agile, con una pelliccia maculata.
    \item Dante perde la speranza poiché la Lonza non si sposta e lo impedisce.
    \item Il tempo mattutino figura anche l'inizio della vita e ciò induce alla speranza.
    \item La speranza viene rovesciata dall'apparizione del Leone, simbolo della superbia.
    \item Il Leone faceva così paura che anche l'aria tremava al suo cospetto.
    \item Senza neanche lasciare il tempo a Dante di accorgersi del Leone che spunta la Lupa.
    \item La Lupa rappresenta l'avidità insaziabile.
    \item Seconda similitudine: come l'avaro che si dispera dopo aver perso tutto, Dante si dispera di aver perso la speranza dell'altezza.
    \item La Lupa rende Dante costretto all'indietreggiare, simboleggiando il ritorno verso la Selva.
    \item Mentre che Dante "cadeva" verso la Selva, vede qualcuno con una voce leggera (si intende che era morto).
    \item Dante urla "Miserere di me", chiedendo pietà.
    \item La figura si presenta, spiegando di essere stato un poeta di nome Virgilio.
    \item Virgilio descrive la cupidigia (la Lupa) e del perché sia insuperabile.
    \item Virgilio spiega che tanti sono caduti nel peccato e profetizza l'arrivo del Veltro che sconfiggerà la Lupa.
    \item Il Veltro non avrà fame di beni materiali ma si ciberà di sapienza, amore e virtù.
    \item Virgilio spiega che il Veltro sarà la salvezza dell'Italia e sconfiggerà la Lupa.
    \item Virgilio si offre di guidare Dante per uscire dalla Selva e tornare in superficie.
    \item Virgilio descrive l'Inferno, dove Dante sentirà disperate strida e vedrà spiriti dolenti.
    \item Virgilio spiega il Purgatorio, dove le anime sono contente di soffrire poiché sperano nel Paradiso.
    \item Virgilio illustra il Paradiso, spiegando che non sarà lui a guidare Dante ma un'anima più degna.
    \item Virgilio spiega che non può andare in Paradiso perché non ha conosciuto Cristo.
    \item Dante accetta Virgilio come guida.
    \item Dante accetta di seguire Virgilio fino al Purgatorio e all'Inferno.
    \item Virgilio inizia a camminare e Dante lo segue.
\end{enumerate}

\newpage
\subsubsection{Inferno, Canto III}
\paragraph*{Antinferno: i Pusillanime} \phantom{}\\
\poetry{
    & \textit{"Per me si va ne la città dolente,}\\
    & \textit{per me si va ne l'etterno dolore,}\\
    3. &\textit{per me si va tra la perduta gente.}\\\\
    & \textit{Giustizia mosse il mio alto fattore;}\\
    & \textit{fecemi la divina podestate,}\\
    6. &\textit{la somma sapïenza e 'l primo amore.}\\\\
    & \textit{Dinanzi a me non fuor cose create}\\
    & \textit{se non etterne, e io etterna duro.}\\
    9. &\textit{Lasciate ogne speranza, voi ch'intrate."}
}
\begin{enumerate}
    \item Questa è l'incisione scritta sopra la porta, la quale nei primi tre versi parla in prima persona con 
    un'anafora "per me si va"; per me si intende "passando dentro di me".
    \item La prima terzina parla del Dolore.
    \item La seconda terzina parla di chi ha creato la porta e dunque anche l'Inferno, ossia dalla trinità, da Dio.
          Questo significa che tutto quello che c'è all'Inferno è giusto e indiscutibile, è motivato dalla volontà di Dio.
    \item La terza terzina parla dell'inizio dell'Inferno, che prima c'erano cose eterne come la materia e i cieli, ma dall'inizio dell'Inferno la sua esistenza sarà a sua volta eterna, dunque chi ci entra dovrà scontarci le pene senza possibilità di uscirne.
    \item 
    \cpoetry{
    Queste parole di colore oscuro \\ 
    vid'ïo scritte al sommo d'una porta
    }
    Dante spiega che le tre terzine precedenti erano incise in colore scuro sopra la porta e che il senso di quelle scritte lo spaventava e faceva fatica a reggere la paura del loro significato.
    \item 
    \cpoetry{
    Ed elli a me, come persona accorta: \\ 
    "Qui si convien lasciare ogne sospetto; \\ 
    ogne viltà convien che qui sia morta.
    }
    Virgilio dice a Dante che deve smettere di esitare e che è il caso di procedere. 
    Dante è un uomo con le sue debolezze ma necessita umanamente di un uomo che lo aiuti con la sua paura.
    \item 
    \cpoetry{
    Noi siam venuti al loco ov'i' t'ho detto \\ 
    che tu vedrai le genti dolorose \\ 
    c'hanno perduto il ben de l'intelletto.
    }
    Virgilio dice a Dante che stanno entrando in un posto dove le persone hanno perso Dio.
    \item 
    \cpoetry{
    E poi che la sua mano a la mia puose \\ 
    con lieto volto, ond'io mi confortai, \\ 
    mi mise dentro a le segrete cose.
    }
    Virgilio prende per mano Dante guardandolo in modo rassicurante, facendolo confortare e varcando la porta. Questo è l'ultimo verso alla luce del sole.
    \item 
    \cpoetry{
    Quivi sospiri, pianti e alti guai \\ 
    risonavan per l'aere sanza stelle, \\ 
    per ch'io al cominciar ne lagrimai.
    }
    Climax di dolore: sospiri, pianti e alti guai. Dante, dopo aver sentito questi lamenti in un luogo senza Dio, piange, provando compassione per tutto il suo percorso nell'Inferno.
    \item 
    \cpoetry{
    Diverse lingue, orribili favelle, \\ 
    parole di dolore, accenti d'ira, \\ 
    voci alte e fioche, e suon di man con elle
    
    facevano un tumulto, il qual s'aggira \\ 
    sempre in quell'aura sanza tempo tinta, \\ 
    come la rena quando turbo spira.
    }
    Anticlimax della precisione delle parole: lingue, favelle, versi, suoni vocali. Dante sente corpi e mani che picchiano tra di loro, tutto mescolato nel buio come un turbine vorticoso di vento e sabbia.
    \item 
    \cpoetry{
    E io ch'avea d'error la testa cinta, \\ 
    dissi: "Maestro, che è quel ch'i' odo? \\ 
    e che gent'è che par nel duol sì vinta?"
    }
    Dante, con il suo pensiero errato, chiede a Virgilio cosa fosse quello che sentiva e chi fossero quelle persone sconfitte dal dolore.
    \item Virgilio risponde che questi erano i dannati che non avevano né fatto del male né del bene, ossia i pusillanimi, gli ignavi.
    \item Il peccato dei pusillanime è rifiutare la loro identità più profonda di uomo, ossia vivere come un animale e rinunciare alla propria essenza e intimità da uomo, ossia vivere senza libero arbitrio.
    \item 
    \cpoetry{
    Mischiate sono a quel cattivo coro \\ 
    de li angeli che non furon ribelli \\ 
    né fur fedeli a Dio, ma per sé fuoro.
    }
    Assieme ai pusillanimi ci sono anche gli angeli che, durante la ribellione di Lucifero a Dio, non si schierarono.
    \item 
    \cpoetry{
    Caccianli i ciel per non esser men belli, \\ 
    né lo profondo inferno li riceve, \\ 
    ch'alcuna gloria i rei avrebber d'elli.
    }
    Gli angeli cacciati dal cielo non sono ricevuti dall'Inferno profondo, poiché i dannati potrebbero vantarsi di condividere la pena con un angelo.
    \item Virgilio nomina tre volte il perché le persone si trovano lì: 
    \cpoetry{
    "che visser sanza 'nfamia e sanza lodo," \\
    "non furon ribelli né fur fedeli a Dio" \\
    "Caccianli i ciel […] né lo profondo inferno"
    }
    \item 
    \cpoetry{
    E io: "Maestro, che è tanto greve \\ 
    a lor che lamentar li fa sì forte?" \\ 
    Rispuose "Dicerolti molto breve,
    
    Questi non hanno speranza di morte, \\ 
    e la lor cieca vita è tanto bassa, \\ 
    che 'nvidïosi son d'ogne altra sorte.
    }
    Dante chiede il perché le persone si lamentano. Virgilio risponde in breve che tutte le anime sperano di essere annichilite per non soffrire più.
    \item 
    \cpoetry{
    Fama di loro il mondo esser non lassa; \\ 
    misericordia e giustizia li sdegna: \\ 
    non ragioniam di lor, ma guarda e passa.
    }
    Virgilio dice a Dante di non parlare neanche dei pusillanime, poiché non meritano parole, e di guardare e andare avanti.
    \item 
    \cpoetry{
    E io, che riguardai, vidi una 'nsegna \\ 
    che girando correva tanto ratta, \\ 
    che d'ogne posa mi parea indegna;
    
    e dietro la venìa sì lunga tratta \\ 
    di gente, ch'i' non avrei creduto \\ 
    che morte tanta n'avesse disfatta.
    }
    Dante vede una bandiera seguita da tantissime persone. La colpa dei pusillanimi è non essersi mai schierati; la pena è seguire per sempre il vessillo (simbolo di schieramento).
    \item 
    \cpoetry{
    Poscia ch'io v'ebbi alcun riconosciuto, \\ 
    vidi e conobbi l'ombra di colui \\ 
    che fece per viltade il gran rifiuto.
    }
    Dante riconosce uno in particolare che fece il suo rifiuto per viltà, mancanza di coraggio. L'ipotesi più accreditata è Celestino V, che lasciò la carica di Papa.
    \item 
    \cpoetry{
    Incontanente intesi e certo fui \\ 
    che questa era la setta d'i cattivi, \\ 
    a Dio spiacenti e a' nemici sui.
    }
    Dante, riconoscendo (per ipotesi) Celestino V, crede che in quel girone ci siano i pusillanimi.
    \item 
    \cpoetry{
    Questi sciaurati, che mai non fur vivi, \\ 
    erano ignudi e stimolati molto \\ 
    da mosconi e da vespe ch'eran ivi.
    }
    I dannati erano nudi e punti con insistenza in faccia da mosconi e vespe, facendo sanguinare il volto; il sangue mescolato con le lacrime viene risucchiato da vermi ai loro piedi.
    \item Dante non si ferma a parlare di loro e se ne va.
\end{enumerate}
\paragraph*{Vero Inferno} \phantom{}\\
\begin{enumerate}
    \item 
    \cpoetry{
    E poi ch'a riguardar oltre mi diedi, \\ 
    vidi genti a la riva d'un gran fiume; \\ 
    per ch'io dissi: "Maestro, or mi concedi

    ch'i' sappia quali sono, e qual costume \\ 
    le fa di trapassar parer sì pronte, \\ 
    com'i' discerno per lo fioco lume."
    }
    Dante vede un fiume e delle persone, e chiede a Virgilio: Chi sono? Quale legge le fa sembrare desiderose di attraversare il fiume?
    \item 
    \cpoetry{
    Ed elli a me: "Le cose ti fier conte \\ 
    quando noi fermerem li nostri passi \\ 
    su la trista riviera d'Acheronte."
    }
    Virgilio risponde che glielo dirà quando arriveranno lì, al fiume Acheronte.
    \item 
    \cpoetry{
    Allor con li occhi vergognosi e bassi, \\ 
    temendo no 'l mio dir li fosse grave, \\ 
    infino al fiume del parlar mi trassi.
    }
    Dante imbarazzato e credendo di essere stato inopportuno con quella domanda, abbassa gli occhi e non parla più fino all'arrivo al fiume.
    \item 
    \cpoetry{
    Ed ecco verso noi venir per nave \\ 
    un vecchio, bianco per antico pelo, \\ 
    gridando: "Guai a voi, anime prave!

    Non isperate mai veder lo cielo: \\ 
    i' vegno per menarvi a l'altra riva \\ 
    ne le tenebre etterne, in caldo e 'n gelo.
    }
    Arrivati in riva al fiume, incontrano Caronte, il traghettatore di anime.
    \item 
    \cpoetry{
    E tu che se' costì, anima viva, \\ 
    pàrtiti da cotesti che son morti". \\ 
    Ma poi che vide ch'io non mi partiva,

    disse: "Per altra via, per altri porti \\ 
    verrai a piaggia, non qui, per passare: \\ 
    più lieve legno convien che ti porti.
    }
    Caronte vede Dante e gli comanda di allontanarsi dalle anime morte poiché lui è una "anima viva", intendendo sia che Dante è ancora vivo, sia perché è un'anima già salva.
    \item 
    \cpoetry{
    E 'l duca lui: "Caron, non ti crucciare: \\ 
    vuolsi così colà dove si puote \\ 
    ciò che si vuole, e più non dimandare."
    }
    Virgilio dice a Caronte di lasciarlo passare, poiché il viaggio è voluto dal cielo.
    \item 
    \cpoetry{
    Quinci fuor quete le lanose gote \\ 
    al nocchier de la livida palude, \\ 
    che 'ntorno a li occhi avea di fiamme rote.
    }
    Descrizione di Caronte: con le guance piene di barba e ruote di fuoco attorno agli occhi.
    \item 
    \cpoetry{
    Ma quell'anime, ch'eran lasse e nude, \\ 
    cangiar colore e dibattero i denti, \\ 
    ratto che 'nteser le parole crude.
    }
    Le anime che hanno sentito tutto il dibattito e soprattutto le parole iniziali, sbiancano dalla paura dopo che hanno capito che le parole erano per loro.
    \item 
    \cpoetry{
    Bestemmiavano Dio e lor parenti, \\ 
    l'umana spezie e 'l loco e 'l tempo e 'l seme \\ 
    di lor semenza e di lor nascimenti.
    }
    Le anime maledicono Dio, i loro genitori e qualsiasi altra cosa che li ha portati in vita: se non fossi mai nato, non avrei peccato e non sarei qui.
    \item Le anime si mettono tutte vicine, piangendo, pronte ad andare dall'altra parte.
    \item 
    \cpoetry{
    Caron dimonio, occhi di bragia, \\ 
    loro accennando, tutte le raccoglie; \\ 
    batte col remo qualunque s'adagia.
    }
    Caronte raggruppa le anime davanti alla sua nave. Le anime stanche che provano a mettersi comodi vengono prese a remate da Caronte.
    \item 
    \cpoetry{
    Come d'autunno si levan le foglie \\ 
    l'una appresso de l'altra, fin che 'l ramo \\ 
    vede a la terra tutte le sue spoglie,

    similmente il mal seme d'Adamo \\ 
    gittansi di quel lito ad una ad una, \\ 
    per cenni come augel per suo richiamo.
    }
    Come un ramo in autunno vede cadere tutte le sue foglie, i dannati salgono sulla nave, lasciando spoglia la riva del fiume.
    \item 
    \cpoetry{
    Così sen vanno su per l'onda bruna, \\ 
    e avanti che sien di là discese, \\ 
    anche di qua nuova schiera s'auna.
    }
    Caronte non fa a tempo a portare le persone dall'altra parte che alla riva si ricrea una folla per salire alla prossima.
    \item 
    \cpoetry{
    Figliuol mio", disse 'l maestro cortese, \\ 
    "quelli che muoion nell'ira di Dio \\ 
    tutti convegnon qui d'ogne paese;
    }
    Virgilio risponde alla prima domanda di Dante: Chi sono? Tutti i dannati che muoiono in tutto il mondo si ritrovano lì.
    \item Virgilio risponde alla seconda domanda di Dante: Perché vogliono andare dall'altra parte?
    \cpoetry{
    e pronti sono a trapassar lo rio, \\ 
    ché la divina giustizia li sprona, \\ 
    sì che la tema si volve in disio.
    }
    Il richiamo di Caronte li spinge ad andare verso il fiume, e la loro paura, avendo coscienza dei loro peccati, diventa desiderio di arrivare nel loro girone.
    \item 
    \cpoetry{
    Quinci non passa mai anima buona; \\ 
    e però, se Caron di te si lagna, \\ 
    ben puoi sapere omai che 'l suo dir suona"
    }
    Virgilio dice a Dante che se Caronte si arrabbia perché non vuole lasciar passare Dante, significa che lui non passerà mai dall'Inferno.
    \item 
    \cpoetry{
    Finito questo, la buia campagna \\ 
    tremò sì forte, che de lo spavento \\ 
    la mente di sudore ancor mi bagna.
    }
    In quel momento viene un terremoto e Dante, parlando al passato, dice che a ripensare a quel terremoto ancora trema e suda dalla paura.
    \item 
    \cpoetry{
    La terra lagrimosa diede vento, \\ 
    che balenò una luce vermiglia \\ 
    la qual mi vinse ciascun sentimento; \\ 
    e caddi come l'uom cui sonno piglia.
    }
    In sequenza del terremoto, il vento e la forte luce, Dante sviene.
\end{enumerate}

\subsubsection{Confronto di Caronte nell'Eneide e nella Commedia}
\begin{enumerate}
    \item Cominciando con le analogie, la prima è data dai capelli bianchi dei due personaggi (vv. 299-300; v. 83).
    \item Per l'aspetto fisico ci sono gli occhi rosso fuoco: per Virgilio sono solo occhi di fiamme, mentre per Dante non sono solo occhi di fiamme ma anche occhi di brace, usando la sua immaginazione.
    \item Entrambi i Caronte sono traghettatori di anime verso l'Inferno (etterne tenebre) e svolgono lo stesso ruolo nelle due opere: oltre a traghettare le anime, impediscono il passaggio dei protagonisti vivi che cercano di passare.
    \item Virgilio ha descritto meglio Caronte, parlando del suo mantello e altre caratteristiche fisiche. Inoltre, descrive meglio la barca sulla quale vengono traghettate le anime.
    \item Si nota come Virgilio tenda a utilizzare molti aggettivi per caratterizzare tutto, mentre Dante fa prevalere i verbi al posto degli aggettivi, cosa caratteristica della sua penna.
    \item Il Caronte dantesco è più aggressivo e urla, mettendo più timore e facendo frasi più brevi ma più d'impatto. Al contrario, quello virgiliano utilizza frasi più solenni e avvolgenti.
    \item Le entrate in scena sono diverse: nella Commedia Caronte irrompe violentemente nella scena, mentre non c'è traccia di un'entrata nell'Eneide.
    \item Caronte viene visto da Virgilio come un Dio, mentre Dante lo vede come un uomo indemoniato.
    \item Entrambi sbarrano la strada ai due protagonisti, ma la motivazione nella Commedia è che Dante è un'anima viva, sia fisicamente viva che già salva, e non passerà mai dall'Inferno il giorno della sua morte fisica. La motivazione nell'Eneide è che la nave può trasportare solo i morti, quindi Enea, essendo vivo, non può passare.
\end{enumerate}

































































































































\end{document}