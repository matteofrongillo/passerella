\documentclass{article}
\usepackage{frongillo}

\title{\textbf{Italiano\\Corso Passerella}}
\author{Matteo Frongillo}

\begin{document}

\maketitle
\tableofcontents

\newpage
\section{Dante}

\subsection{La Commedia}

\subsubsection{Canto I}

\poetry{Nel mezzo del cammin di nostra vita}

L'età di vita media nel Medioevo, come dato Biblico e Filosofico, era considerata di 70 anni.\\
Con questa coordinata di tempo abbiamo un'età di 35 anni che ci porta a fissare una data di inizio, ossia nel 1300 (Dante nacque nel 1265).\\
(Nell'Inferno XXI vv. 112-114, si deduce che la data d'inizio sia veramente il 1300, poiché rammenta come sono passati 1266 anni dalla morte di Cristo, morto nel 34).\\
Dante sceglie l'anno del primo Giubileo della storia (la purificazione dei peccati), durante la settimana santa del 1300.\\
Usando ``nostra vita", l'autore amplia la prospettiva su tutta l'umanità, ossia vuole rappresentare TUTTE le persone.

\poetry{mi ritrovai per una selva oscura,\\
che la diritta via era smarrita.}

La ``selva oscura" rappresenta per lui un periodo di peccato di Dante, nella quale non entra la Luce Divina.\\
Dante spiega come abbia smarrito la ``diritta via" ossia la Luce, la via verso Dio.\\
La ``via era smarrita" poiché smarrita significa che può essere ritrovata.

\poetry{Ahi quanto a dir qual era è cosa dura\\
esta selva selvaggia e aspra e forte\\
che nel pensier rinova la paura!}

Dante usa questa terzina per mettere in chiaro che il viaggio è già stato fatto ed è chiuso, infatti dice che era difficile ripensare alla Selva.\\
I tre aggettivi della Selva sono: Selvaggia = Disumana, nel senso che non c'è uomo che vi ci viva; Aspra = Intricata, fitta; Forte = Difficile da cui è difficile uscire.\\
Gli aggettivi sono disposti in Climax ascendente.

\poetry{Tant’è amara che poco è più morte;}

Dante compara la Selva alla morte, dicendo che la Selva era poco meno spaventosa della morte stessa.

\poetry{ma per trattar del ben ch'i' vi trovai,\\
dirò de l'altre cose ch'i' v'ho scorte.}

Sì che la Selva era spaventosa, disumana e intricata ma Dante vuole parlare comunque di quello che trova al suo interno.\\
Al ``ben" si riferisce a ciò che di buono Dante trova, dunque Virgilio.\\
Al ``l'altre cose" si riferisce a tutto ciò che non era buono, dunque le Fiere.\\
Quando Dante scrive ``al vi / v'ho" non si riferisce però alla Selva, ma a TUTTO il viaggio stesso, dunque traslando il significato del ``ben" e ``de l'altre" ampliando la prospettiva, dunque tutti i beni e tutti i non-beni che in tutta la Commedia Dante incontra e si cimenta.\\
Con questa prospettiva ampliata si può dunque chiudere il capitolo della Vita Nova sicché la privatissima vita amorosa di Dante viene ampliata con tutti con conoscenza di Dio e tutto lo spazio percorso e gli incontri legati ad esso.

\poetry{Io non so ben ridir com'i' v'intrai\\
tant'era pien di sonno a quel punto\\
che la verace via abbandonai.}

La ``verace via" indica la retta via.\\
Il ``sonno" intende la pigrizia, poiché abbandonando il bene la coscienza viene abbandonata.

\poetry{Ma poi ch'i' fui al piè d'un colle giunto,\\
là dove terminava quella valle\\
che m'avea di paura il cor compunto,\\
guardai in alto e vidi le sue spalle\\
vestite già de' raggi del pianeta\\
che mena dritto altrui per ogne calle.}

Dante ALZA GLI OCCHI, gesto simbolico che indica lui che guarda verso la salvezza.\\
Dante con ``raggi del pianeta" indica il Sole, il pianeta che indica la retta via, il bene, Dio.\\
Selva e Sole sono due opposti: in basso sta l'oscurità, in alto la luce, in basso il peccato, in alto la salvezza.\\
Con questa figura vengono prefigurati i tre regni del libro. La selva è un bosco associato all'Inferno. Il colle associato al Purgatorio, mentre il Sole/La cima del Colle è associato al Paradiso.

\poetry{Allor fu la paura un poco queta,\\
che nel lago del cor m'era durata\\
la notte ch'i' passai con tanta pieta.}

Dante impaurito scioglie la paura alzando gli occhi al Sole, così da avere speranza.

\poetry{E come quei che con lena affannata,\\
uscito fuor del pelago a la riva,\\
si volge a l'acqua perigliosa e guata,\\
così l'animo mio, ch'ancor fuggiva,\\
si volse a retro a rimirar lo passo\\
che non lasciò già mai persona viva.}

La prima similitudine: Così come un naufrago che dopo aver combattuto le acque impaurito, una volta arrivato a riva si volta a guardare il mare, il quale non ha mai lasciato nessuno in vita. Dunque, Dante scappando e scampando dalla Selva si rigira a guardare quel posto spaventoso, dicendo che chi persevera nel peccato è destinato a morire.\\
Guatare=Guardare con partecipazione, spesso paura.

\poetry{Poi ch'èi posato un poco il corpo lasso,\\
ripresi via per la piaggia diserta,\\
sì che 'l piè fermo sempre era 'l più basso.}

Dante riprende il viaggio, iniziando a camminare verso il pendio.\\
Con il piede fermo più basso intende che, oltre a essere in salita, indica che il piede più saldo, dunque quello fermo, è quello verso il basso, dunque quello più vicino alla Selva, come se lui volesse andarsene ma il piede fosse sempre saldo verso il basso.

\poetry{Ed ecco, quasi al cominciar de l'erta,\\
una lonza leggera e presta molto,\\
che di pel macolato era coverta;}

Ed ecco: Espressione avversativa di Dante.\\
Dante incontra la Lonza: rapida, agile e con una pelliccia maculata.\\
A queste caratteristiche viene associata al peccato di lussuria.\\
La lussuria è basata molto sull'aspetto e la Lonza si presentava affascinante, con un pelo elegante.\\
Lussuria=Vizio inteso come l'abbandono alle proprie passioni o anche a divertimenti di natura generica, senza il controllo da parte della nostra ragione e della nostra morale.

\poetry{e non mi si partia dinanzi al volto,\\
anzi 'mpediva tanto il mio cammino,\\
ch'i' fui per ritornar più volte vòlto.}

La Lonza non si sposta in fronte a Dante, e lui tanto che lei impediva il passaggio, lo fece voltare indietro più volte verso la Selva. Questo porta Dante a perdere la speranza.

\poetry{Temp'era dal principio del mattino,\\
e'l sol montava 'n sù con quelle stelle\\
ch'eran con lui quando l'amor divino\\
mosse di prima quelle cose belle;}

Dante dice che è mattina, secondo la genesi la Terra è nata di primavera, allora è una mattina di primavera.\\
Il tempo mattutino figura anche l'inizio della vita e ciò induce alla speranza.\\
Torna la speranza, che aveva perso dalla Lonza.

\poetry{sì ch'a bene sperar m'era cagione\\
di quella fiera a la gaetta pelle\\
l'ora del tempo e la dolce stagione;}

Dante è pieno di speranza che con l'inizio del giorno e soprattutto di primavera sicuramente poteva dimenticarsi in della spaventosa Lonza incontrata precedentemente.

\poetry{ma non sì che paura non mi desse\\
la vista che m'apparve d'un leone.}

``Ma" avversativo: Rovescia la speranza con la nuova perdita della speranza.\\
Non appena Dante smette di temere la Lonza gli appare un Leone.

\poetry{Questi paera che contra me venisse\\
con la test'alta e con rabbiosa fame,\\
sì che parea che l'aere ne tremesse.}

Il Leone va in contro a Dante.\\
Il Leone rappresenta la superbia ``con la test'alta" ossia bramoso.\\
Il verso ``parea che l'aere ne tremesse": Il leone faceva così paura che anche l'aria tremava al suo cospetto.\\
Superbia=Radicata convinzione della propria superiorità (reale o presunto) che si traduce in atteggiamenti di orgoglioso distacco o anche di ostentato disprezzo verso gli altri.

\poetry{Ed una lupa, che di tutte brame\\
sembiava carca ne la sua magrezza,\\
e molte genti fé già viver grame,}

Senza neanche lasciare il tempo a Dante di accorgersi del Leone che spunta la Lupa.\\
La Lupa rappresenta il peccato più grave dei tre: l'avidità insaziabile.\\
Avarizia=Brama di possessi materialistici.

\poetry{questa mi porse tanto di gravezza\\
con la paura ch'uscia di sua vista,\\
ch'io perdei la speranza de l'altezza.}

La Lupa, carica di tutte le bramosie umane, rende Dante costretto all'indietreggiare, facendolo praticamente scivolare verso il basso ``mi porse tanto di gravezza", letteralmente rende più pendente il suolo.\\
Questa terzina simboleggia il ritorno verso la Selva.

\poetry{E qual è quei che volontieri acquista,\\
e giunge 'l tempo che perder lo face,\\
che 'n tutti suoi pensier piange e s'attrista;}

Seconda similitudine: come l'avaro (o un giocatore d'azzardo) che di fatto si dispera dopo aver perso tutto, poiché la sua felicità stava nei beni materiali, Dante si dispera di aver perso la speranza dell'altezza.

\poetry{tal mi fece la bestia sanza pace,\\
che, venendomi 'ncontro, a poco a poco\\
mi ripigneva là dove 'l sol tace.}

Fine della similitudine: Come Dante era sicuro di poter salire sul colle, si dispera dopo che la Lupa gli pare insuperabile.\\
Spiega poi come la Lupa, andandogli contro, lo indietreggia verso la Selva.

\poetry{Mentre ch'i' rovinava in basso loco,\\
dinanzi a li occhi mi si fu offerto\\
chi per lungo silenzio parea fioco.}

Mentre che Dante ``cadeva" verso il luogo in basso (la Selva), davanti ai suoi occhi vede qualcuno che per tanto tempo di silenzio aveva una voce leggera (si intende che era morto).

\poetry{Quando vidi costui nel gran diserto,\\
"Miserere di me", gridai a lui,\\
"qual che tu sii, od ombra od omo certo!"}

Dante da solo nella Selva urla al qualcuno ``Abbi pietà di me, che tu sia vivo o morto!".

\poetry{Rispuosemi: "Non omo, omo già fui,\\
e li parenti miei furon lombardi,\\
mantoani per patrïa ambedui.}

La figura gli chiarisce che non è più vivo e si presenta con la sua provenienza geografica, soprattutto questo avviene per appartenenza politica.\\
Legge di Tobler-Mussafier: Rispuosemi \textrightarrow\  Mi rispuose.

\poetry{Nacqui sub Iulio, ancor che fosse tardi,\\
e vissi a Roma sotto 'l buono Augusto\\
nel tempo de li dèi falsi e bugiardi.}

Indicazione di tempo, la figura spiega che è nato quando ancora Giulio Cesare era vivo, specificando che, essendo Giulio morto quando la figura era giovane, ha vissuto sotto Augusto, nell'epoca degli dèi falsi e bugiardi, ossia nel paganesimo, Avanti Cristo.

\poetry{Poeta fui, e cantai di quel giusto\\
figliuol d'Anchise che venne di Troia,\\
poi che 'l superbo Ilïon fu combusto.}

Indica l'attività di questa figura: Era poeta.\\
Una perifrasi per spiegare che la figura ha celebrato Enea, (quel giusto figliuol d'Anchise), dopo che la rocca di Troia venne bruciata.\\
Questa perifrasi è un chiaro riferimento all'Eneide, scritto da Virgilio, ed fa intuire che Dante ha letto l'Opera.\\
Da questi versi si evince come il salvatore di Dante nella selva sia un poeta, a delineare l'importanza della poesia e che per Dante la poesia è la salvezza.

\poetry{Ma tu perché ritorni a tanta noia?\\
perché non sali il dilettoso monte\\
ch'è principio e cagion di tutta gioia?"}

La figura chiede a Dante perché torna verso la ``noia" (al tormento) e del perché Dante stia indietreggiando verso il peccato e non andando verso il colle, ossia verso la salvezza.\\
Ci si aspetterebbe che Dante rispondesse alla domanda della figura, ma lui riconoscendo il poeta lo elogia.

\poetry{"Or se' tu quel Virgilio e quella fonte\\
che spandi di parlar sì largo fiume?"\\
rispuos'io lui con vergognosa fronte.\\
"O de li altri poeti onore e lume,\\
vagliami 'l lungo studio e 'l grande amore\\
che m'ha fatto cercar lo tuo volume.\\
Tu se' lo mio maestro e 'l mio autore,\\
tu se' solo colui da cu' io tolsi\\
lo bello stilo che m'ha fatto onore.}

Dante stupito riconosce Virgilio e lo elogia.\\
``rispuos'io lui con vergognosa fronte." significa che Dante lo elogia con rispetto e riverenza.\\
``O de li altri poeti onore e lume," significa che tutti i poeti contemporanei di Dante onorano Virgilio e che lo trovano una musa ispiratrice.\\
``vagliami 'l lungo studio e 'l grande amore | che m'ha fatto cercar lo tuo volume." Qui Dante dichiara un debito poetico a Virgilio perché essendo che Dante ha studiato intensamente e con amore l'Eneide e le altre opere di Virgilio allora Virgilio gli deve un favore.\\
``lo bello stilo che m'ha fatto onore." Dante dichiara un debito stilistico ``lo stile alto" che Dante ha utilizzato in altre canzoni dottrinate che ha scritto in passato prima della Commedia, stile acquisito dallo studio intenso di Virgilio.

\poetry{Vedi la bestia per cu' io mi volsi;\\
aiutami da lei, famoso saggio,\\
ch'ella mi fa tremar le vene e i polsi"}

Dante risponde alla domanda di Virgilio sul perché lui non stesse salendo sul colle.\\
Inoltre Dante fa la richiesta a Virgilio di aiutarlo a salire sul colle.

\poetry{"A te convien tenere altro vïaggio",\\
rispuose, poi che lagrimar mi vide,\\
"se vuo' campar d'esto loco selvaggio;}

Virgilio dice a Dante, che nel mentre piange, DEVE fare un altro viaggio per arrivare alla salvezza.\\
Questa terzina è quella più importante di tutto il primo Canto.

\poetry{ché questa bestia, per la qual tu gride,\\
non lascia altrui passar per la sua via,\\
ma tanto lo 'mpedisce che l'uccide;}

Virgilio spiega a Dante che la Lupa è insuperabile, tutti sono impediti a passare e chi indugia al fuggire verso il colle finisce per venir ucciso.\\
È inoltre un'allegoria al peccato, che andando verso peccato più forte si rischia di morire peccatori e che nessuno riesce a sfuggire ad esso.

\poetry{e ha natura sì malvagia e ria,\\
che mai non empie la bramosa voglia,\\
e dopo 'l pasto ha più fame che pria.}

Virgilio descrive la cupidigia (la Lupa) e del perché sia insuperabile.\\
La Lupa dopo aver mangiato ha ancora più fame di prima. Questo rappresenta anche il peccato di cupidigia.\\
Cupidigia=Desiderio intenso e infermabile; ha sempre bisogno di più.

\poetry{Molti son li animali a cui s'ammoglia,\\
e più saranno ancora, infin che 'l veltro\\
verrà, che la farà morir con doglia.}

Virgilio parla con linguaggio profetico (una profezia, scritta con verbi al futuro).\\
Tante sono le persone che cadono nel peccato e più ne saranno in futuro e verrà sconfitta dal Veltro.\\
Il Veltro è un cane da caccia, un animale più forte della Lupa.\\
Essendo una profezia detta da Virgilio, si possono solo fare ipotesi su chi sia il Veltro.

\poetry{Questi non ciberà terra né peltro,\\
ma sapïenza, amore e virtute,\\
e sua nazion sarà tra feltro e feltro.}

Il Veltro sarà qualcuno che non avrà fame di terra (espansionismo) né peltro (denaro, ricchezza), ossia con valori antitetici e senza fame di beni materiali. Il Veltro dunque si ciberà di sapienza, amore e virtù, che sono i tre attributi della trinità.\\
La sua nascita (nazion) sarà tra feltro e feltro.

Il feltro ha varie interpretazioni:
- feltro = cielo \textrightarrow\  personaggio provvidenziale, ossia qualcuno mandato dal cielo\\
- feltro = panno vile \textrightarrow\  un francescano, qualcuno che fa della rinuncia e umiltà la sua ragione di vita\\
- tra feltro e feltro = tra Feltre e Montefeltro \textrightarrow\  Cangrande della Scala, un personaggio nato a Verona (tra i due feltri) che ha ospitato Dante durante il suo esilio, al quale dedica il Paradiso nella Commedia.

La tesi più accreditata è vedere il feltro come il panno che avvolgeva i bossoli (contenitori delle urne) delle elezioni imperiali, dunque sarà un imperatore eletto dal popolo che rimetta il popolo sulla retta via, lontani dai peccati.\\
Dante poneva molta fiducia nell'imperatore Arrigo VII e probabilmente intendeva lui.

\poetry{Di quella umile Italia fia salute\\
per cui morì la vergine Cammilla,\\
Eurialo e Turno e Niso di ferute.}

Virgilio spiega che il Veltro sarà la salvezza di quella umile Italia (quella nata nello scontro tra troiani e latini) per cui morirono Cammilla, Eurialo, Turno e Niso per ferite.

\poetry{Questi la caccerà per ogne villa,\\
fin che l'avrà rimessa ne lo 'nferno,\\
là onde 'nvidia prima dipartilla.}

La conclusione della profezia della sconfitta della Lupa da ``questi", ossia il Veltro, che la rimanderà nell'Inferno dopo che per invidia nei confronti di Dio, il Demonio la scagliò fuori per farla diffondere nell'umanità.

\poetry{Ond'io per lo tuo me' penso e discerno\\
che tu mi segui, e io sarò la tua guida,}

Virgilio valuta e si mette a disposizione per fare da guida a Dante per uscire dalla Selva e tornare in superfice.

\poetry{e trarrotti di qui per loco etterno;\\
ove udirai le disperate strida,\\
vedrai li antichi spiriti dolenti,\\
che la seconda morte ciascun grida;}

Virgilio spiega a Dante dove lo porta, ossia l'Inferno, e:
Cosa sentirà \textrightarrow\  Disperate strida\\
Cosa vedrà \textrightarrow\  Spiriti dolenti\\
Spiega che vedrà chi si pente di essere morto per non morire anche con l'anima, coloro che sono anche consapevoli che la loro pena sarà eterna.

\poetry{e vedrai color che son contenti\\
nel foco, perché speran di venire\\
quando che sia a le beate genti.}

Virgilio spiega il purgatorio, dove le anime sono contente di soffrire poiché sono consapevoli e certe che la loro sofferenza terminerà e avranno l'accesso al Paradiso.

\poetry{A le quai poi se tu vorrai salire,\\
anima fia a ciò più di me degna:\\
con lei ti lascerò nel mio partire;}

Virgilio illustra tutte le tappe del viaggio di Dante, qui spiega il Paradiso.\\
Spiega che se Dante vorrà poi salire al Paradiso, non sarà lui a guidarlo ma un'anima più degna di lui.\\
Virgilio dà l'opzione a Dante: SE vuole salire, poiché alla fine del purgatorio Dante è già salvo e purificato.\\
L'accesso al Paradiso è un'opzione ma il passaggio nell'Inferno e nel Purgatorio è obbligatorio per essere puri da ogni peccato.

\poetry{ché quello imperador là sù regna,\\
perch'i' fu' ribellante a la sua legge,\\
non vuol che 'n sua città per me si vegna.}

Virgilio spiega che lui, non avendo fatto il bene in vita (essendo che non ha mai conosciuto Cristo e dunque Dio), è un dannato per sempre e dunque non può salire nel Suo regno, dunque in Paradiso.

\poetry{In tutte le parti impera e quivi regge;\\
quivi è la sua città e l'alto seggio:\\
oh felice colui cu' ivi elegge!"}

Virgilio spiega il suo pentimento nel sapere che non andrà mai in Paradiso e un po' invidia chi ci può andare.

\poetry{E io a lui: "Poeta, io ti richeggio\\
per quello Dio che tu non conoscesti,\\
acciò ch'io fugga questo male e peggio,}

Dante accetta Virgilio come guida.\\
A differenza di Virgilio, Dante può pronunciare il nome di Dio. Nel nome di Dio che purtroppo Virgilio non potrà conoscere.

\poetry{che tu meni là dov'or dicesti,\\
sì ch'io veggia la porta di san Pietro\\
e color cui tu fai cotanto mesti".}

Dante accetta Virgilio come guida fino al purgatorio e anche l'Inferno.\\
Questa terzina ripercorre in ordine opposto i Regni che visiterà (o che ha visitato, visto che la narrazione è al passato).

\poetry{Allor si mosse, e io li tenni dietro.}

Virgilio inizia a camminare e Dante lo segue.\\
Come il canto comincia con un cammino metaforico, viene terminato con un cammino fisico.

\newpage
\section{Ariosto}

\subsection{Biografia}

Ariosto viveva a Ferrara in una modesta casa con sua moglie.\\
Intraprende una vita piena di medietà.

\subsubsection{Orlando Furioso}
È il continuo del libro \textit{Orlando innamorato} di Este.\\
La metrica dell'opera è ABABABCC.\\
Sono tutte ottave (8 versi per strofa).\\
Nel canto I, le prime 4 ottave fanno da proemio all'opera,\\
le ottave 5 e 9 sono di richiamo a ciò che succede nell'\textit{Orlando innamorato} mentre dall'ottava 10 in poi è pura invenzione.\\
Orlando è stato uno dei più grandi paladini di Carlo Magno.\\
Il libro venne pubblicato nel 1516. Vengono stampate all'epoca 1300 in totale. Oggi ne rimangono solo 12, 6 di quelle in Italia.

\begin{tabular}{|c|c|}
\hline
Anno & Descrizione \\
\hline
1516 & 40 canti \\
1521 & 40 canti \\
1525 & \textit{Prose della volgar lingua} \\
1531 & 46 canti \\
\hline
\end{tabular}

Nella copia del 1516 erano presenti 40 canti, così come nella ristampa del 1521.\\
Nel 1525 escono le \textit{Prose della volgar lingua} di Pietro Bembo (chi scrive in prosa scriva come Boccaccio, chi scrive in poesia scriva come Petrarca).\\
Dopo questa pubblicazione allora Ariosto riscrive tutta l'opera adeguandola a un italiano toscaneggiante come imposto da Bembo.\\
Nel 1532 viene dunque pubblicata la nuova versione aggiornata, la quale presentava 46 canti.

\subsubsection{Apprezzamento dell'opera}
\begin{enumerate}
    \item Isabella d'Este, Lettera a Ippolito d'Este (sorella di Ariosto)\\
    Scrive una lettera dove ringrazia il fratello per avergli fatto leggere una parte della divertente opera che Ariosto stava scrivendo. Significa che nel 1507 l'Orlando fosse già in produzione o fisicamente o concettualmente nella testa dell'autore.
    \item Machiavelli, Lettera a Lodovico Alamanni\\
    Nella lettera Machiavelli scrive di quanto il libro sia bello alla lettura.\\
    Però si lamenta di essere stato lasciato in disparte dalla lista di Ariosto dei suoi autori preferiti, dicendo che avrebbe potuto anche fargli un riconoscimento.
\end{enumerate}

\subsubsection{Struttura del testo}
\textbf{Proemio} \phantom{}

\begin{tabular}{|c|c|}
\hline
Sezione & Capitoli \\
\hline
Protasi & 1 - 2.4 (v.2 r.4) \\
Invocazione & 2.5 - 2.8 \\
Dedica & 3 - 4 \\
\hline
\end{tabular}

\begin{tabular}{|c|c|}
\hline
Capitoli & Descrizione \\
\hline
1 - 4 & Proemio \\
5 - 9 & ``gionta'', di transizione \\
10 $>$ & Narrazione di pura invenzione \\
\hline
\end{tabular}

\subsection{Orlando Furioso}

\subsubsection{Canto I - vv. 1-32}







\end{document}