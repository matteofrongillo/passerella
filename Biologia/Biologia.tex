\documentclass{article}

\usepackage[fleqn]{amsmath}
\usepackage{amssymb}
\usepackage{hyperref}
\usepackage{url}
\usepackage{graphicx}
\usepackage{geometry}
\usepackage[italian]{babel}
\usepackage{enumitem}
\usepackage{parskip}
\usepackage{chemfig}
\usepackage{pdfpages}
\usepackage{xcolor}
\usepackage{tikz}
\usepackage{fancybox}
\usepackage{makecell}
\usepackage{soul}
\usepackage{sankey}

\geometry{
    a4paper,
    total={170mm, 257mm},
    left=20mm,
    top=20mm
}

\hypersetup{
    colorlinks=true,
    linkcolor=black,
    urlcolor=blue,
    pdftitle={Chimica}
}

% === COMMANDS ===
\newcommand{\figbox}[1]{
    \begin{figure*}[h!]
        \begin{center}
            \fbox{#1}
        \end{center}
    \end{figure*}
}

\newcommand*\circled[1]{\tikz[baseline=(char.base)]{
            \node[shape=circle,draw,inner sep=1.1pt] (char) {#1};}
}
\newcommand\angstrom{\mbox{\normalfont\AA}}
\newcommand\namebond[4][5pt]{\chemmove{\path(#2)--(#3)node[midway,sloped,yshift=#1]{#4};}}
\newcommand\arcbetweennodes[3]{
    \pgfmathanglebetweenpoints{\pgfpointanchor{#1}{center}}{\pgfpointanchor{#2}{center}}
    \let#3\pgfmathresult
}
\newcommand\arclabel[8][stealth-stealth,shorten <=1pt,shorten >=1pt]{
    \chemmove{
        \arcbetweennodes{#4}{#3}\anglestart \arcbetweennodes{#4}{#5}\angleend
        \draw[#1]([shift=(\anglestart:#2)]#4)arc(\anglestart:\angleend:#2);
        \pgfmathparse{(\anglestart+\angleend)/2}\let\anglestart\pgfmathresult
        \node[shift=(\anglestart:#2+1pt)#4,anchor=\anglestart+180,rotate=\anglestart#7,
        inner sep=0pt,outer sep=#8]at(#4){#6};
    }
}

% === TEXT ===
\title{\textbf{Biologia \\ Passerella 2023-24}}
\author{Matteo Frongillo}

\begin{document}

\maketitle
\tableofcontents
\pagebreak

\section{Gli organismi}

\section{Sistema vivente}
La classificazione di sistema vivente è suddivisa in due macrocategorie:
\begin{itemize}
    \item \textbf{Visione meccanicistica}: Un sistema vivente è riconosciuto come tale
        se possiede questi elementi:
        \begin{itemize}
            \item Basi cellulari;
            \item Ha una forma e una funzione;
            \item Codice genetico (XNA);
            \item Scambio di materia e di energia;
            \item Ciclo vitale e riproduzione;
            \item Reazione agli stimoli;
            \item Omeostasi;
            \item Evoluzione e varietà.
        \end{itemize}
    \item \textbf{Visione sistemica}: Un sistema vivente è riconosciuto come tale
        se possiede questi elementi:
        \begin{itemize}
            \item Schema autopoietico;
            \item Struttura dissipativa;
            \item Processo cognitivo.
        \end{itemize}
\end{itemize}

\subsection{Sistema autopoietico}
Il sistema autopoietico è un ciclo logico composto da componenti molecolari,
da un sistema confinato e da reti di reazioni metaboliche.\\
Il sistema autoconfinato rappresenta l'entità che mantiere la propria struttura
attraverso la continua rigenerazione e ristrutturazopme dei suoi componenti.
... ... ...


\subsection{Le caratteristiche}
Un organismo è un essere vivente che possiede le seguenti caratteristiche:
\begin{itemize}
    \item Si nutre
\end{itemize}





\pagebreak
\begin{tikzpicture}
    \begin{sankeydiagram}[start style=arrow,end style=arrow]
        \sankeynodestart{quantity=15, name=node1, at={0,0}}
        \node[left=0.5cm, above, rotate=90] (label1) {CH$_2$O + O$_2$};
        
        \sankeyadvance{node1}{2cm}
        \sankeyfork{node1}{5/node2, 10/node3}

        \sankeyturn{node2}{90}
        \sankeyadvance{node2}{.2cm}
        \sankeyend{node2}
        \node[above=2.1cm, right=2cm] (label2) {Energia};

        \sankeyturn{node3}{0}
        \sankeyadvance{node3}{1cm}
        \sankeyend{node3}
        \node[above=-0.26cm, right=3.5cm] (label2) {CO$_2$ + H$_2$O};
    \end{sankeydiagram}.
\end{tikzpicture}

    
    

\end{document}